\section{\texorpdfstring{Definition of \mttwo}{Definition of MT2}}
\label{sect:mt2def}
The $\mttwo$ variable~\cite{Lester:1999tx,Barr:2003rg} is used in this analysis to discriminate between the SUSY signal and the SM backgrounds as proposed in Ref. \cite{Barr:2009wu}. The variable was introduced to measure the mass of primary pair-produced particles, decaying eventually to undetected particles (e.g. \PSGczDo). Assuming the two primary supersymmetric particles undergo the same decay chain with visible and undetectable particles in the final state, the system can be described by the visible mass ($\mvisi$), transverse energy ($\etvisi$), and transverse momentum ($\vptvisi$) of each branch ($i=1,2$), together with the 
missing transverse momentum (\ptvecmiss) which is shared between the two decay chains. The \ptvecmiss is interpreted as the sum of the transverse momenta
of the neutralinos, $\vec{p}_{\rm T}^{\PSGczDo(i)}$.
In practice, in decay chains with neutrinos, \ptvecmiss also includes contributions from the $\pt$ of the neutrinos.

The transverse mass of each branch can be defined as 
\begin{linenomath}
\begin{equation}
\label{eq:mtdef}
(\mt^{(i)})^{2}= (\mvisi)^2+m^2_{\PSGczDo}+2(\etvisi\et^{\PSGczDo(i)}-{\vptvisi}.\,{\vec{\pt}^{\PSGczDo(i)}}).
\end{equation}
\end{linenomath}

\noindent Using the correct neutralino mass, this distribution has an endpoint at the mass of the primary particle~\cite{Affolder:2000bpa,Abazov:2002bu}. 
% similar to the W boson transverse mass used to measure $m_{\rm W}$
%As a generalization of the transverse mass, the $\mttwo$ variable is proposed to overcome the problem of unknown $\pt^{\PSGczDo(i)}$. The kinematic endpoint of $\mttwo$ carries model independent information about the mass difference between the primary and the secondary particles. 
For a given $m_{\PSGczDo}$, the $\mttwo$ variable is defined as
\begin{linenomath}
\begin{equation}
\label{eq:mt2def}
\mttwo(m_{\PSGczDo})= \min_{\vec{p}_{\rm T}^{\PSGczDo(1)}+\vec{p}_{\rm T}^{\PSGczDo(2)}=\ptvecmiss}\,\left[\,\max\,\{ \, \mt^{(1)},\,\mt^{(2)}\,\}\,\right].
\end{equation}
\end{linenomath}

For the correct value of $m_{\PSGczDo}$, the kinematic endpoint of the $\mttwo$ distribution is at the mass of the primary particle, and it shifts accordingly when the assumed $m_{\PSGczDo}$ is lower or higher than the correct value. In this analysis, 
the visible part of the decay chain consists of either the two hadronically decaying tau leptons (\tauTau channel)
or a combination of a muon or an electron with a \Tau candidate ($\leptonTau$ channel), so $\mvisi$ is the mass of a lepton and can be set to zero. We also set $m_{\PSGczDo}=0$. 

With  our choices of $m_{\PSGczDo}$ and $\mvisi$, the resulting \mttwo value is close to zero, regardless of the values of \MPT and the \PT of 
the tau candidates, for the 
back-to-back topology of \tauTau or \leptonTau  
events that would be produced in Drell-Yan events or in di-jets events if both jets are misidentified. This is not the case for signal events where the taus or leptons are generally not in a back-to-back topology due 
to the presence of two undetected neutralinos.

The distribution of \mttwo reflects the scale of the produced particles and is much higher for heavy sparticles
compared to the lighter SM particles. Hence, SUSY 
could manifest itself
as an excess of events in the high-side tail of the \mttwo distribution.
% It was shown previously \cite{Khachatryan:2014qwa} and \cite{Chatrchyan:2012jx}    
% that \mttwo is a powerful variable to search for SUSY in both leptonic and hadronic final states.
