\section{Conclusions}
\label{sect:conclusion}
A search for SUSY in the $\tau\tau$ final state was performed where the
$\tau$ pair could be produced in a cascade decay from the electroweak production of a chargino pair.  The data analyzed were from proton-proton collisions
%electroweak production of \PSGcpDo pair. in proton-proton collisions 
at $\sqrt{s}$ = 8\TeV collected by the CMS detector at the LHC and corresponding to a\
n integrated luminosity between 18.1 and $19.6~\mathrm{fb}^{-1}$.%, collected by the CMS detector.
To maximize the sensitivity, event selections are optimized for \tauTau (small $\Delta$m), 
\tauTau (large $\Delta$m) and \leptonTau channels using the variables \mttwo, \tauMT, and \SumMT.
The observed number of events is consistent with the SM expectations. 
%We have used different channels and search regions to increase the sensitivity to
%different regions of phase space. All channels considered have at least one hadronic $\tau$ decay.
% \mttwo of two leptons is used as a search variable to
%distinguish between signal and background. In a special part of the phase space with
%a moderate \mttwo, the sum of the transverse mass of two $\tau$ leptons was found to be a
%useful variable.
%There is no excess of events with respect to the SM expectations.
%The other channels were investigated, but they do not add any axclusion power to the analysis.
%Backgrounds and their systematic uncertainties are discussed in details. 
%The expected exclusion limits are also presented for different combination of the channels.
In the context of simplified models, charginos lighter than 421\GeV 
for a massless neutralino are excluded at a 95\% confidence level.
%The upper limits for the direct stau pair production are also provided, 
Upper limits on the direct $\tilde{\tau}\tilde{\tau}$ production cross section are provided, but the limits are more than three times
larger than the theoretical NLO cross sections, 
%region of \sTau mass can not be excluded 
even for a massless neutralino.

