\section{The CMS detector and event reconstruction}
\label{sect:CMSRec}
The central feature of the CMS apparatus is a superconducting solenoid of 6\unit{m} internal diameter, providing a magnetic field of 3.8\unit{T}. Within the superconducting solenoid volume are a silicon pixel and strip tracker, a lead tungstate crystal electromagnetic calorimeter (ECAL), and a brass and scintillator hadron calorimeter (HCAL), each composed of a barrel and two endcap sections. Muons are measured in gas-ionization detectors embedded in the steel flux-return yoke outside the solenoid. Extensive forward calorimetry complements the coverage provided by the barrel and endcap detectors. 
A more detailed description of the CMS detector, together with a definition of the coordinate system used and the relevant kinematic variables, can be found in Ref. \cite{Chatrchyan:2008zzk}.
%For completeness, a short review is given here. 


Events from pp interactions must satisfy the requirements of a two-level trigger system.
The first level (L1) of the CMS trigger system, composed of custom hardware processors, uses information from the calorimeters and muon detectors to select the most interesting events in a fixed time interval of less than 4\mus. The high-level trigger (HLT) processor farm further decreases the event rate from around 100\unit{kHz} to around 400\unit{Hz}, before data storage. 

The particle-flow event algorithm~\cite{CMS-PAS-PFT-09-001,CMS-PAS-PFT-10-001} reconstructs and identifies each individual particle with an optimized combination of information from the various elements of the CMS detector. 
%The energy of photons is directly obtained from the ECAL measurement, corrected for zero-suppression effects. The energy of electrons is determined from a combination of the electron momentum at the primary interaction vertex as determined by the tracker, the energy of the corresponding ECAL cluster, and the energy sum of all bremsstrahlung photons spatially compatible with originating from the electron track. The energy of muons is obtained from the curvature of the corresponding track. The energy of charged hadrons is determined from a combination of their momentum measured in the tracker and the matching ECAL and HCAL energy deposits, corrected for zero-suppression effects and for the response function of the calorimeters to hadronic showers. Finally, the energy of neutral hadrons is obtained from the corresponding corrected ECAL and HCAL energy. 
Jets are reconstructed from the particle-flow objects with the anti-$k_t$ clustering
algorithm~\cite{Cacciari:2008gp} with a distance parameter of 0.5. We apply
\pt- and $\eta$-dependent corrections to account for residual
effects of non-uniform detector response~\cite{Chatrchyan:2011ds}.
A correction to account for multiple pp collisions within the same or a nearby
bunch crossing (pileup interactions) is estimated on an event-by-event basis using the
jet-area method described in Ref.~\cite{Cacciari:2007fd}, and is
subtracted from the reconstructed jet \pt.
The combined secondary vertex algorithm is used to identify (``b-tag'') jets 
originating from b quarks.  This algorithm 
 is based on the reconstruction of secondary vertices, together with track-based lifetime information~\cite{Chatrchyan:2012jua}. 
%In this analysis the "medium" working point is used. 
%The working point corresponds to an average b-tagged jets efficiency of 70\%, 
In this analysis a working point is chosen such that, for jets with a \PT value greater than 60\GeV, the efficiency for tagging a jet containing a b quark is 70\%, with a light-quark jet misidentification rate of 1.5\%, and $\cPqc$ quark jet misidentification rate of 20\%.
Scale factors are applied to the simulated events to reproduce the tagging efficiencies measured in the data, 
separately for jets originating from b/$\cPqc$ quarks or from light quarks.
Jets with  \PT $>$ 40\GeV and $\abs{\eta} < 5.0$ and b-tagged jets with \PT $>$ 20\GeV and $\abs{\eta} < 2.4$ are considered in this analysis.


The particle-flow objects are used to reconstruct the missing transverse momentum  vector \ptvecmiss, defined as the negative of the vector sum of the transverse momenta of all reconstructed particles.  Corrections are applied to ensure consistency between
\ptvecmiss and the corrections to jet energies described above.  The missing transverse momentum in the event (\MPT) is defined as the magnitude of \ptvecmiss.

Hadronically-decaying $\tau$ leptons, referred to as \Tau, are reconstructed using the hadron-plus-strips algorithm~\cite{Khachatryan:2015dfa}.
The constituents of the reconstructed jets are used to identify individual tau decay modes with one charged 
hadron and up to two neutral pions, or three charged hadrons. 
Additional discriminators are used to separate \Tau from electrons and muons.
Prompt $\tau$ leptons are expected to be isolated in the detector.
To discriminate them from QCD jets, we use a measure of isolation 
based on the charged hadrons and photons falling within 
a cone around the tau momentum direction after correcting for the effect of
pileup \cite{Khachatryan:2014wca}. The ``loose'', ``medium'' and ``tight'' working points are defined
by requiring the measure of isolation not to exceed thresholds of 2.0, 1.0,
and 0.8 GeV, respectively.
 A similar isolation algorithm is 
used in this analysis to separate leptons (e or $\mu$) from tau decays from 
those arising from hadron decays within jets.

\section{Monte Carlo samples}
\label{sect:MCSamples}
Events from SM processes that may represent significant sources of backgrounds, $\cPZ$+jets, \wjets, $\cPqt\cPaqt$, and di-boson 
are generated using the \MADGRAPH 5.1~\cite{Alwall:2011uj} generator. 
Single top quark and Higgs boson events are generated by {\POWHEG} 1.0~\cite{Nason:2004rx,Frixione:2007vw,Alioli:2009je,Alioli:2010xd}.
In the following figures and tables, the events containing at least one top quark or one $\cPZ$ boson are referred to as ``Top'' and ``ZX'', respectively. 
Events from Higgs boson production via gluon fusion, vector boson fusion or in association with a $\cPZ$ or $\PW$ boson or a \ttbar pair are referred to as ``Higgs''. The masses of the top quark and Higgs boson are set to be 172.5\GeV and 125\GeV, respectively.

%The simplified model which is used to describe the signal events is shown in Fig.~\ref{fig:Productions} (left). 
In signal samples, a pair of charginos (\chione) 
are produced and decay exclusively to the final states that contain two tau leptons, two tau neutrinos, and %$\tau$, two $\nu_{\tau}$ and 
two neutralinos (\PSGczDo) as shown in Fig.~\ref{fig:Productions} (left). 
The mediators in the decay of the \chione can be either a \sTau or a $\sNu_{\tau}$. 
The masses of the \sTau and the $\sNu_{\tau}$ are set to be equal and at the mean value of the \chione and \PSGczDo masses. 
Thus they are produced on-shell.
The two distinct decay chains in the left diagram of Fig.~\ref{fig:Productions} 
are assumed to have equal branching fractions of 50\%. 
For parton shower and fragmentation, all generators are interfaced with \PYTHIA 6.4~\cite{Sjostrand:2006za}.
\PYTHIA is also used to generate signal events (chargino pair production). To improve the modeling of $\Pgt$ decays, 
we use the \TAUOLA 1.1.1a~\cite{Davidson:2010rw} package. 


In the data set considered in this paper,
there were on average 21 proton-proton interactions (``pileup'') in each bunch crossing.
Additional interactions are generated with \PYTHIA and superimposed on simulated events in a manner consistent with the
luminosity profile of the data set.
The detector response in the Monte Carlo background event samples is modeled by a
detailed simulation
of the CMS detector based on {\GEANTfour}~\cite{Agostinelli:2002hh}.  
On the other hand, in order to reduce  computational requirements, signal events 
are processed by the CMS fast simulation \cite{Abdullin:2011zz} instead of {\GEANTfour}. 
All simulated events are reconstructed with the same algorithms as collision data.

The SM backgrounds are normalized using the most accurate calculations of the cross sections available 
in the literature. These cross sections correspond to next-to-next-to-leading order (NNLO) accuracy for $\cPZ$+jets~\cite{Melnikov:2006kv} 
and \wjets~\cite{xsec_WZ} events. For the $\cPqt\cPaqt$ simulated samples, the cross section used is calculated to full NNLO accuracy including
%The cross section of $\cPqt\cPaqt$ simulated sample at full NNLO accuracy including 
the resummation of next-to-next-to-leading-logarithmic (NNLL) terms~\cite{Czakon:2011xx}. %is used~\cite{Czakon:2011xx}. 
The event yields from di-boson production are normalized to the next-to-leading order (NLO) cross section  taken from Ref.~\cite{Campbell:2011bn}. 
The \textsc{Resummino}~\cite{Fuks:2012qx,Fuks:2013vua,Fuks:2013lya} calculations at NLO+NLL are used to calculate the signal cross sections, where 
NLL refers to next-to-leading-logarithmic precision.
