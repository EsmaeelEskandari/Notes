\section{Additional information for new model testing}%Information to test new models} %Greg%
\label{sect:model}
In the previous sections, a simplified SUSY model was used to optimize the selections and interpret the results. 
Here, the main efficiencies are reported versus generated values, so that these results can be used in an approximate manner to examine new models in a MC generator-level study. %Greg%
%that can be used to examine the new models approximately in a Monte Carlo generator-level study. 
The number of the remaining signal events and its uncertainty that %Greg%
can be evaluated by a generator-level study 
should be combined statistically with the results in Table \ref{tbl:yieldSysSummary} to find the upper limit 
on the number of the signal events
and decide if a model is excluded or still allowed according to  the analysis presented in the paper.

Efficiencies are provided against the kinematic properties (e.g., \pt) of visible $\tau$ lepton decay products at th generator level. The visible $\tau$ lepton (\visTau), if it decays leptonically, is defined as the 4-vector of the lepton. In hadronic decays, \visTau is the difference between the 4-vector of the $\tau$ lepton and neutrino in the hadronic decay.%Greg% hadronic decays, the difference between the 4-vector of $\tau$ and its neutrino is attributed to the visible $\tau$. %It is, hereafter, referred to as \visTau.
The visible $\tau$ objects are required to pass the offline kinematic selections ($\eta$ and \pt requirements). The \genMET variable is defined as the magnitude of the negative vector sum of \visTau pairs in the transverse plane. The 4-vector of the \visTau objects and \genMET are used to calculate the transverse mass (\mt) of the \visTau objects and  also the generator-level \mttwo.  %Greg% also the generated \mttwo. 
All efficiencies are derived using the SUSY chargino pair production sample. 
%Greg% in our SUSY signal sample, which is a chargino pair production sample. 
The chargino mass is varied from 120 to 500 \GeV and the neutralino mass is varied from 1 to 500 \GeV.
%Greg% The chargino mass goes from 120 to 500 \GeV and the neutralino mass goes from 1 to 500 \GeV. 
Table \ref{tbl:EffTauLep}
\begin{table}[!htb]
\begin{center}
\caption{Efficiencies to select a lepton or \Tau in different channels. Here, $\Tau^1$ and $\Tau^2$ stand for leading and subleading (in pT) \Tau in the \tauTau channel.} 
%Greg% $\Tau^1$ and $\Tau^2$ stand for leading and next-to-leading \Tau in the \tauTau channel.}
\begin{tabular}{|c|c|c|c|c|c|}
\hline
\pt(\visTau) (\GeV)       & $e$ for $e\Tau$ & $\mu$ for $\mu\Tau$  & \Tau for $\ell\Tau$    &  $\Tau^1$ for \tauTau & $\Tau^2$ for \tauTau\\
\hline\hline
%0-10                      &    0.15       &    0.01              &         0.001          &       0.0             & 0.52 \\\hline
%10-20                     &    0.14       &    0.72              &         0.004          &       0.0             & 0.54 \\\hline
20--30                     &    0.27       &    0.80              &         0.16           &       0.0             & 0.00 \\\hline
30--40                     &    0.68       &    0.86              &         0.29           &       0.0             & 0.00 \\\hline
40--60                     &    0.75       &    0.87              &         0.34           &       0.03            & 0.61 \\\hline
60--80                     &    0.80       &    0.89              &         0.38           &       0.10            & 0.69 \\\hline
80--120                    &    0.83       &    0.90              &         0.40           &       0.18            & 0.70 \\\hline
120--160                   &    0.86       &    0.90              &         0.41           &       0.22            & 0.70 \\\hline
160--200                   &    0.87       &    0.91              &         0.41           &       0.24            & 0.71 \\\hline
$>$ 200                   &    0.89       &    0.92              &         0.41           &       0.26            & 0.71 \\\hline

\end{tabular}
\label{tbl:EffTauLep}
\end{center}
\end{table}
shows the efficiencies of selecting a lepton or \Tau for different channels versus \pt(\visTau). 
These efficiencies include the scale factors, and efficiencies of object identification, isolation and trigger.
%object identification and isolation and trigger also.
%The \visMET variable is defined as the magnitude of the negative vector sum of \visTau pairs in the transverse plane. 
Table \ref{tbl:EffMet}
\begin{table}[!htb]
\begin{center}
\caption{Efficiency to pass the \MPT requirement in different channels versus \genMET.}
\begin{tabular}{|c|c|}
\hline
\genMET  (\GeV)        & all channels\\
\hline\hline
0--10                   &    0.52 \\\hline
10--20                  &    0.58 \\\hline
20--30                  &    0.68 \\\hline
30--40                  &    0.79 \\\hline
40--50                  &    0.87 \\\hline
50--60                  &    0.93 \\\hline
60--70                  &    0.95 \\\hline
70--80                  &    0.97 \\\hline
80--90                  &    0.98 \\\hline
90--100                 &    0.98 \\\hline
100--120                &    0.99 \\\hline
120--140                &    0.99 \\\hline
140--160                &    0.99 \\\hline
$>$160                 &    1.0  \\\hline

\end{tabular}
\label{tbl:EffMet}
\end{center}
\end{table}
shows the efficiency in different channels to pass the \MPT $>$ 30 \GeV requirement as a function of the \genMET. 
%The mass of the system of the selected pair is used to parameterize 
%the efficiency to pass the cuts on the reconstructed invariant mass. 
Table \ref{tbl:EffMass}
\begin{table}[!htb]
\begin{center}
\caption{Efficiency to pass the invariant mass requirements in different channels versus generated mass.}
\begin{tabular}{|c|c|c|}
\hline
generated mass (\GeV)  & $\ell\Tau$  &  \tauTau \\
\hline\hline
0--5                  &    0.00     &   0.00   \\\hline
5--10                 &    0.10     &   0.00   \\\hline
10--15                &    0.23     &   0.20   \\\hline
15--20                &    0.97     &   0.90   \\\hline
20--25                &    0.99     &   0.94   \\\hline
25--30                &    1.00     &   0.98   \\\hline
30--35                &    0.99     &   1.00   \\\hline
35--40                &    0.98     &   1.00   \\\hline
40--45                &    0.84     &   0.99   \\\hline
45--50                &    0.16     &   0.95   \\\hline
50--55                &    0.04     &   0.68   \\\hline
55--60                &    0.02     &   0.18   \\\hline
60--65                &    0.01     &   0.06   \\\hline
65--70                &    0.04     &   0.03   \\\hline
70--75                &    0.23     &   0.05   \\\hline
75--80                &    0.78     &   0.15   \\\hline
80--85                &    0.91     &   0.40   \\\hline
85--90                &    0.96     &   0.78   \\\hline
90--95                &    0.97     &   0.92   \\\hline
95--100               &    0.98     &   0.95   \\\hline
100--105              &    1.00     &   0.98   \\\hline
105--110              &    1.00     &   0.99   \\\hline
$>$ 110              &    1.00     &   1.00   \\\hline

\end{tabular}
\label{tbl:EffMass}
\end{center}
\end{table}
shows the efficiency in different channels to pass the requirement of the reconstructed invariant mass 
versus the invariant mass of the 
\visTau pair (generated mass). The requirements
on the invariant mass of the reconstructed pair are ($>$ 15 \GeV) and ($<$ 45 or $>$ 75 \GeV) for the $\ell\Tau$ channels 
and ($<$ 55 or $>$ 85 \GeV) for the \tauTau channel. 
The efficiency to pass the (\mttwo $>$ 90 \GeV) requirement in $\ell\Tau$ signal region and \tauTau \binone is shown in Table \ref{tbl:EffMT2}. 
\begin{table}[!htb]
\begin{center}
\caption{Efficiency to pass the \mttwo $>$ 90 \GeV requirement in different channels versus generated \mttwo.}
\begin{tabular}{|c|c|c|}
\hline
generated \mttwo (\GeV)    & $\ell\Tau$  &  \tauTau \binone \\
\hline\hline
0--20                     &    0.00     &   0.00  \\\hline
20--40                    &    0.002    &   0.01  \\\hline
40--50                    &    0.01     &   0.01  \\\hline
50--60                    &    0.02     &   0.03  \\\hline
60--70                    &    0.05     &   0.07  \\\hline
70--80                    &    0.13     &   0.17  \\\hline
80--90                    &    0.35     &   0.44  \\\hline
90--100                   &    0.65     &   0.73  \\\hline
100--110                  &    0.82     &   0.88  \\\hline
110--120                  &    0.90     &   0.94  \\\hline
120--130                  &    0.93     &   0.97  \\\hline
130--140                  &    0.95     &   0.98  \\\hline
140--160                  &    0.96     &   0.98  \\\hline
160--180                  &    0.97     &   0.99  \\\hline
$>$ 180                  &    0.97     &   1.00  \\\hline

\end{tabular}
\label{tbl:EffMT2}
\end{center}
\end{table}
Table \ref{tbl:EffTauMT}
\begin{table}[!htb]
\begin{center}
\caption{Efficiency to pass the \tauMT requirement in $\ell\Tau$ channels versus generated \tauMT.}
\begin{tabular}{|c|c|}
\hline
generated \tauMT (\GeV)  & $\ell\Tau$ \\
\hline\hline
100--125                  &   0.01   \\\hline
125--150                  &   0.03   \\\hline
150--170                  &   0.09   \\\hline
170--190                  &   0.26   \\\hline
190--200                  &   0.51   \\\hline
200--210                  &   0.67   \\\hline
210--230                  &   0.82   \\\hline
230--250                  &   0.91   \\\hline
250--275                  &   0.94   \\\hline
275--300                  &   0.97   \\\hline
$>$ 300                  &   1.00   \\\hline

\end{tabular}
\label{tbl:EffTauMT}
\end{center}
\end{table}
shows the efficiency in the $\ell\Tau$ channels to pass the \tauMT $>$ 200 \GeV requirement versus generated \tauMT.


In the \tauTau \bintwo, the reconstructed \mttwo is constrained between 40 and 90 \GeV. Table \ref{tbl:EffMT2SR2}
\begin{table}[!htb]
\begin{center}
\caption{Efficiency to pass the \mttwo requirement in \tauTau \bintwo versus generated \mttwo.}
\begin{tabular}{|c|c|}
\hline\hline
generated \mttwo (\GeV)  &  \tauTau \bintwo \\
\hline
0--20     & 	0.08  \\\hline
20--40    & 	0.43  \\\hline
40--50    & 	0.75  \\\hline
50--60    & 	0.82  \\\hline
60--70    & 	0.81  \\\hline
70--80    & 	0.72  \\\hline
80--90    & 	0.49  \\\hline
90--100   & 	0.24  \\\hline
100--110  & 	0.11  \\\hline
110--120  & 	0.05  \\\hline
120--130  & 	0.03  \\\hline
130--140  & 	0.02  \\\hline
140--160  & 	0.01  \\\hline
160--180  & 	0.01  \\\hline
$>$ 180  & 	0.00  \\\hline

\end{tabular}
\label{tbl:EffMT2SR2}
\end{center}
\end{table}
shows the efficiency in \tauTau \bintwo to pass the 40 $<$ \mttwo $<$ 90 \GeV requirement versus generated \mttwo. 
The last selection in this channel is
the requirement on \SumMT which is calculated using the 4-vector of the two \visTau and \genMET. Table \ref{tbl:EffSumMT} 
\begin{table}[!htb]
\begin{center}
\caption{Efficiency to pass the \SumMT requirement in \tauTau \bintwo versus generated \SumMT.}
\begin{tabular}{|c|c|c|}
\hline
generated \SumMT (\GeV)  &  \tauTau \bintwo\\
\hline\hline 
$<$ 80       &  0.00  \\\hline
80--180       &  0.16  \\\hline
180--200      &  0.19  \\\hline
200--210      &  0.25  \\\hline
210--220      &  0.30  \\\hline
220--230      &  0.36  \\\hline
230--240      &  0.43  \\\hline
240--250      &  0.52  \\\hline
250--260      &  0.55  \\\hline
260--270      &  0.61  \\\hline
270--280      &  0.67  \\\hline
280--290      &  0.68  \\\hline
290--300      &  0.73  \\\hline
300--320      &  0.76  \\\hline
320--340      &  0.77  \\\hline
340--360      &  0.80  \\\hline
360--380      &  0.81  \\\hline
380--400      &  0.81  \\\hline
$>$ 400      &  0.82  \\\hline

\end{tabular}
\label{tbl:EffSumMT}
\end{center}
\end{table}
shows the efficiency in \tauTau \bintwo to pass the \SumMT $>$ 250 \GeV requirement versus generated \SumMT.

To use these efficiencies, one needs to multiply the values one after another and combine the final value with the values reported in Table \ref{tbl:yieldSysSummary}  statistically, to decide if a signal point is excluded. 
At the generator level, a pair of $\ell\Tau$ or \tauTau is selected, when the \visTau objects pass
the corresponding offline kinematic selections.



The efficiencies were used to reproduce the yields in the SMS plane. The results were in agreement with the yields from the full chain of 
simulation and reconstruction within ~30\%.
A user of these efficiencies should be aware that some assumptions can be
broken close to the diagonal (very low mass difference between chargino and neutralino) and nonphysical estimation 
can be drived by using the reported efficiencies. This region, known as the compressed region requires a separate analysis, 
because the mass difference of mother and daughter is comparable to the kinematic to the kinematic requirements used to select the objects.
