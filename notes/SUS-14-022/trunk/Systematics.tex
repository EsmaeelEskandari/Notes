\section{Systematic uncertainties}
\label{sect:sys}
Systematic uncertainties can affect the shape or normalization of the
backgrounds estimated from Monte Carlo (\ttbar, $Z+$ jets, dibosons and Higgs decays), 
as well as the signal acceptance. 
%To calculate the uncertainties for signal, 3 signal pointsare used which are ($m(\chione) = 180\,\GeV$, $m(\PSGczDo) = 60\,\GeV$), (240, 40) and (380, 1) representing low, moderate and high delta mass respectively.
The uncertainties are listed below and summarized in Table~\ref{Tab.SYS}.
\begin{table}[!h]
\begin{center}
\small{
\begin{tabular}{|l|c|c|c|c|}
\hline\hline
Systematic uncertainty source & \eTau & \muTau & \tauTau SR1 & \tauTau SR2\\
\hline\hline
{Luminosity}&\multicolumn{4}{c|}{$2.6\%$} \\\hline
{$\tau_{had}$ energy scale}&\multicolumn{4}{c|}{$25\%$} \\\hline
{Electron trigger, id, iso efficiency}& $2\%$ & \multicolumn{3}{c|}{} \\\hline
{Muon trigger, id, iso efficiency}& &$2\%$ & \multicolumn{2}{c|}{} \\\hline
{$\tau_{had}$ id efficiency}& \multicolumn{4}{c|}{$6\%$} \\\hline
{$\tau_{had}$ trigger efficiency}& \multicolumn{2}{c|}{$3\%$}&\multicolumn{2}{c|}{$4.5\%$ per leg} \\\hline
{Pile-up}&\multicolumn{4}{c|}{$4\%$} \\\hline
{\MET}&\multicolumn{4}{c|}{$5\%$} \\\hline
{b-jet ID}& $4\%$ & $4\%$ & - & $4\%$ \\\hline
Total(backgrounds) & $26\%$ & $26\%$ & $27\%$  & $27\%$\\\hline
Low rate backgrounds &50\%  & 50\%   & 50\%    & 50\%\\\hline
\multicolumn{5}{|c|}{only for signal} \\\hline
{ISR}&\multicolumn{4}{c|}{$3\%$} \\\hline
{$\mindphifour$}&\multicolumn{4}{c|}{$6\%$} \\\hline
{$\tau_{had}$ energy scale}&\multicolumn{4}{c|}{$15\%$} \\\hline
{PDF}&\multicolumn{4}{c|}{$2\%$} \\\hline
{b-jet ID}& $8\%$ & $8\%$ & - & $8\%$ \\\hline
Total(Signal) & $20\%$ & $20\%$ & $19\%$  & $20\%$\\
\hline
\hline
\end{tabular}
}
\end{center}
\caption{
  Summary of systematic uncertainties.
}
\label{Tab.SYS}
\end{table}


\begin{itemize}

\item The uncertainty on the luminosity  is $2.6\%$ \cite{CMS-PAS-LUM-13-001}.  This affects the
  normalization of all Monte Carlo samples.
 
\item  The energy scales for electron, muon and \Tau objects affect the shape of various kinematical distributions.
 The systematic uncertainties on the muon and electron energy scales are negligible.
The visible energy of \Tau object in the Monte Carlo simulation is scaled up and down
by 3\%, and all \Tau-related variables are recalculated. The resulting variations in
final yields are taken as the systematic uncertainties. They amount to 10\% for both
background and signal events. In the part of the signal phase space which is accessible by
the analysis, the value is almost constant in different points and a conservative value
is selected.


\item The uncertainty on electron and muon trigger, identification, and
  isolation efficiencies is 2\%.

\item The uncertainty on the \Tau identification efficiency is 6\%. 
  The uncertainty on the efficiency of the \Tau leg of the \eTau and
  \muTau (\tauTau) triggers amount to 3.0\% (4.5\% per leg).
  A ``tag-and-probe'' technique on $\cPZ\to \Pgt\Pgt$ events is used to estimate the 
  uncertainties.

\item The uncertainty due to the b-tagged jets scale factor is evaluated by varying the 
factors within their uncertainties. The yields of signal and background events are changed by 8\% 
and 4\%, respectively.
 
\item To evaluate the uncertainty due to pile-up, the assumed inelastic pp cross-section is
  varied by 5\%, resulted in the change in the number of simulated pile-up interactions.
 The relevant acceptances for signal and background events are changed by 4\%.


\item The uncertainty on the signal acceptance due to parton-distribution function uncertainties 
  is taken to be 2\% from a similar analysis \cite{Khachatryan:2014qwa}.

\item The uncertainty on the signal acceptance associated with initial state radiation (ISR)
is evaluated by comparing the efficiencies of jet-related requirements between \PYTHIA
 and \MADGRAPH which is a matrix-element event generator. Using the SM WW process which
 is expected to be similar to chargino pair-production, we assign a 3\% uncertainty on 
the efficiency of  b-tagged jets veto and a 6\% uncertainty on the $\Delta \Phi$ requirement. The ISR
 uncertainty is not considered for the background samples, due to the usage of matrix-
 element event generators.

\item The sources of \MPT uncertainty are the energy scales of lepton, \Tau, and jet
objects and unclustered energy.  The ``unclustered energy'' is for the energy of the reconstructed objects which
 do not belong to any jet or lepton with pT $>$ 10 \GeV. The effect of lepton and \Tau
 energy scales is discussed above. The contribution from the uncertainty of the jet energy scale (2-10\% depending on $\eta$  and \PT) and
 unclustered energy (10\%) is found to be negligible. A conservative value of 5\% uncertainty
 is assigned to both signal and background processes using the Monte
 Carlo simulation studies.

\item The statistics in the simulated Monte Carlo samples are also a
 source of the systematics uncertainties which are taken to be 20\% for the background processes and 10\% for the signal events.

\item The performance of the fast detector simulation is insufficient, compared to the full detector simulation, especially in
 track reconstruction. It can affect the \Tau isolation. A 5\% systematic uncertainty per
 \Tau leg is assigned by comparing the \Tau isolation/identification efficiency in the fast
 and full simulations.

\item For some backgrounds like \ttbar,  dibosons and Higgs decays, the remaining 
events from the simulation are very small. A 50\% uncertainty is considered for these backgrounds to account for the theoretical uncertainty of the
cross section calculation as well as the shape mismodeling.
\end{itemize}


\noindent All systematic uncertainties are added in quadrature. The total uncertainties on the signal acceptance in the \leptonTau and \tauTau 
channels are 20\% and 25\%, respectively; 25\% and 28\% on Monte Carlo predictions for W+jets and DY events in the \leptonTau and \tauTau  channels, respectively.
The uncertainty of the other small backgrounds (Higgs boson, diboson, ttbar production) is 50\% for all channels.


