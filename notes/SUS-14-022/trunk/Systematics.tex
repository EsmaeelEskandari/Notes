\section{Systematic uncertainties}
\label{sect:sys}
Different systematic uncertainties that can affect either shape or normalization of the MC driven backgrounds
and signal are considered and evaluated.
Most of them are found to be negligible. In the following the sources of the uncertainties and the ways that they are evaluated are reported. 

Table \ref{Tab.SYS} summarizes the systematics uncertainties for sigmal and backgrounds.
\begin{table}[!Hhtb]
\begin{center}
\small{
\begin{tabular}{|l|c|c|c|c|}
\hline\hline
Systematic uncertainty source & $e\tau_{had}$ & $\mu\tau_{had}$ & $\tau_{had}\tau_{had}$ binI & $\tau_{had}\tau_{had}$ binII\\
\hline\hline
{Luminosity}&\multicolumn{4}{c|}{$2.6\%$} \\\hline
{$\tau_{had}$ energy scale}&\multicolumn{4}{c|}{$25\%$} \\\hline
{Electron trigger, id, iso efficiency}& $2\%$ & \multicolumn{3}{c|}{} \\\hline
{Muon trigger, id, iso efficiency}& &$2\%$ & \multicolumn{2}{c|}{} \\\hline
{$\tau_{had}$ id efficiency}& \multicolumn{4}{c|}{$6\%$} \\\hline
{$\tau_{had}$ trigger efficiency}& \multicolumn{2}{c|}{$3\%$}&\multicolumn{2}{c|}{$4.5\%$ per leg} \\\hline
{Pile-up}&\multicolumn{4}{c|}{$7\%$} \\\hline
Total(backgrounds) & $27\%$ & $27\%$ & $27\%$  & $27\%$\\\hline
\multicolumn{5}{|c|}{only for signal} \\\hline
{ISR}&\multicolumn{4}{c|}{$3\%$} \\\hline
{$\mindphifour$ cut}&\multicolumn{4}{c|}{$6\%$} \\\hline
{$\tau_{had}$ energy scale}&\multicolumn{4}{c|}{$15\%$} \\\hline
{PDF}&\multicolumn{4}{c|}{$2\%$} \\\hline
Total(Signal) & $20\%$ & $20\%$ & $20\%$  & $20\%$\\
\hline
\hline
\end{tabular}
}
\end{center}
\caption{  Systematic uncertainties results for different channels .}
\label{Tab.SYS}
\end{table}

 Luminosity: The uncertainty on the luminosity  is $2.6\%$ for $2012$ data.
 
 Lepton and $\hadtau$ energy scale: The systematic uncertainty due to muon and electron energy scale are found to be small.
The energy of \hadtau's is scaled up and down by $3\%$ and different \Tau related variables are re-calculated.  The variation in the final yield is considered as the
systematic uncertainty due to the \Tau energy scale. This value is 25\% for background and 15\% for signal.

 Lepton trigger, identification and isolation efficiency: The uncertainties on electron and muon triggers, identification and isolation efficiencies are $2\%$. 
The uncertainty on the $\hadtau$ identification efficiency amounts to $6\%$. 
The uncertainty on the efficiency of the \Tau leg of the $e\hadtau$ and $\mu\hadtau$ ($\tauTau$) triggers amount to $3.0\%$ ($4.5\%$ per leg).

 Pile-up: The minimum bias cross section is varied $5 \%$ up and down. It is found to introduce $~7 \%$ systematic for all channels.    

 PDF: The signal acceptance changes due to PDF uncertainties is expected to be small. 
The amount of this uncertainty is about $2\%$ in the similar analysis \cite{Khachatryan:2014qwa}.

ISR: Since our signal is generated by \PYTHIA that does not  model the initial state radiations (ISR) properly, another MC sample which is similar to our signal, e.g, WW sample 
which is generated by a matrix element generator is used and the efficiency of different jet related cuts is compared between the signal and WW sample. Difference between the
efficiencies is taken as the systematic uncertainty due to mismodelling of ISR. For b-veto 3\% and for cut on   the minimum angle in the transverse plane between 
the \MET and the jets 6\% systematic uncertainties are  calculated.  

All systematics are considered for signal samples, but for the backgrounds, the last two sources are ignored due to using the NLO generators 
for the backgrounds. The dominant source of the background is the $\Tau$ energy scale which is dominated by the lack of the statistics in the 
final remaining MC events, especially because the uncertainty is less for signal samples where the statistics is higher. 
After adding different systematic uncertainties quadratically, 
we assign 20\% and 25\% relative uncertainty on signal and MC driven backgrounds, respectively.

