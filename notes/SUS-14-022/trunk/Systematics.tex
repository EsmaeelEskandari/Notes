\section{Systematic uncertainties}
\label{sect:sys}
Systematic uncertainties can affect the shape or normalization of the
backgrounds estimated from Monte Carlo (\ttbar, $Z+$ jets, dibosons and Higgs decays and \wjets in \tauTau \bintwo), 
as well as the signal acceptance.
These uncertainties are listed below and summarized in Table~\ref{Tab.SYS}.
\begin{table}[!h]
\begin{center}
\small{
\begin{tabular}{|l|c|c|c|c|}
\hline\hline
Systematic uncertainty source & \eTau & \muTau & \tauTau SR1 & \tauTau SR2\\
\hline\hline
{Luminosity}&\multicolumn{4}{c|}{$2.6\%$} \\\hline
{$\tau_{had}$ energy scale}&\multicolumn{4}{c|}{$25\%$} \\\hline
{Electron trigger, id, iso efficiency}& $2\%$ & \multicolumn{3}{c|}{} \\\hline
{Muon trigger, id, iso efficiency}& &$2\%$ & \multicolumn{2}{c|}{} \\\hline
{$\tau_{had}$ id efficiency}& \multicolumn{4}{c|}{$6\%$} \\\hline
{$\tau_{had}$ trigger efficiency}& \multicolumn{2}{c|}{$3\%$}&\multicolumn{2}{c|}{$4.5\%$ per leg} \\\hline
{Pile-up}&\multicolumn{4}{c|}{$4\%$} \\\hline
{\MET}&\multicolumn{4}{c|}{$5\%$} \\\hline
{b-jet ID}& $4\%$ & $4\%$ & - & $4\%$ \\\hline
Total(backgrounds) & $26\%$ & $26\%$ & $27\%$  & $27\%$\\\hline
Low rate backgrounds &50\%  & 50\%   & 50\%    & 50\%\\\hline
\multicolumn{5}{|c|}{only for signal} \\\hline
{ISR}&\multicolumn{4}{c|}{$3\%$} \\\hline
{$\mindphifour$}&\multicolumn{4}{c|}{$6\%$} \\\hline
{$\tau_{had}$ energy scale}&\multicolumn{4}{c|}{$15\%$} \\\hline
{PDF}&\multicolumn{4}{c|}{$2\%$} \\\hline
{b-jet ID}& $8\%$ & $8\%$ & - & $8\%$ \\\hline
Total(Signal) & $20\%$ & $20\%$ & $19\%$  & $20\%$\\
\hline
\hline
\end{tabular}
}
\end{center}
\caption{
  Summary of systematic uncertainties.
}
\label{Tab.SYS}
\end{table}


\begin{itemize}

\item The uncertainty on the luminosity  is $2.6\%$.  This affects the
  normalization of all Monte Carlo samples.
 
\item  The systematic uncertainties on the muon and electron energy scales
  are negligible.  In the case of \Tau, the visible energy in the Monte Carlo
  is scaled up and down by $3\%$, and all \Tau-related variables are
  recalculated.  The resulting variations in final yields are taken as
  systematic
  uncertainties.  They amount to 25\% for backgrounds and 15\% for signal.
  In the part of the signal phase space which is accessible by the analysis,
  the value is almost constant in different points and a conservative value is selected.
 % {\bf (Doesn't the signal uncertainty depend on the signal point?)}

\item The uncertainty on electron and muon trigger, identification, and
  isolation efficiencies is $2\%$.
%  {\bf (Shoudn't you say how you obtained these?)}

\item The uncertainty on the $\hadtau$ identification efficiency is $6\%$. 
  The uncertainty on the efficiency of the \Tau leg of the $e\hadtau$ and
  $\mu\hadtau$ ($\tauTau$) triggers amount to $3.0\%$ ($4.5\%$ per leg).
  A ``tag-and-probe'' technique on $\cPZ\to \Pgt\Pgt$ events is used to estimate the 
  uncertainties.
%  {\bf (Shoudn't you say how you obtained these?)}

\item To evaluate the uncertainty due to pileup, the assumed inelastic
  pp cross-section is varied by 5\%.  This then changes the number
  of simulated pileup interactions, and changes the relevant acceptance
  to the processes of interest by $~4 \%$.

\item The uncertainty on the signal acceptance due to parton-distribution
  function uncertainties is expected to be small.
  It is about $2\%$ in the similar analysis \cite{Khachatryan:2014qwa}.
%  {\bf (I am not sure you can get away with thise statement in a paper!)}

\item The uncertanty on the signal acceptance associated with initial state
  radiation (ISR) is evaluated by comparing the efficiency of jet-related
  requirements between \PYTHIA and \MADGRAPH which is a matrix element generator 
  for the $WW$ SM process which
  is similar to chargino pair-production.  From these studies we estimate
  a 3\% uncertainty on the bveto efficiency and a 6\% uncertainty on the
  $\Delta \Phi$ requirement.
  The ISR uncertainty is not considered for the background samples, due to the
  usage of  matrix element  generators.
%  {\bf (This is not correct.  First of all Madgraph is not NLO, it is
%    only partly NLO.  Second, we have a prescription in the SUSY group on how
%    to reweight and/or assign a systematic due to the modeling of
%    the \ttbar transverse momentum)}


%\noindent {\bf What about the btag uncertainty from the btag group?  What about the uncertainty associated with \MET resolution and tails that would throw some of your SM backgrounds like \ttbar or WW into the MT2 tail?  What about the uncertainties on the normalization of the cross-sections for the backgrounds that you take from Monte Carlo?}
\item The \MET uncertainty receives contribution from different sources. The lepton, \Tau, jet energy scale 
(2–10\% depending on $\eta$ and \PT) and unclustered energy scale (10\%). The ``unclustered energy'' stands for the energy of the 
reconstructed objects which do not belong to a jet or lepton with \PT $>$ 10 \GEV.
The effect of lepton and \Tau energy scale are already discussed. The contribution of the jet energy scale and unclustered energy is 
found to be negligible. A conservative relative uncertainty of 5\% is assigned to the signal and background events which are read from the 
Monte Carlo simulation.
\item For some backgrounds like \ttbar, $Z+$ jets, dibosons and Higgs decays, the remaining 
events from the simulation are very small. A 50\% uncertainty is considered for these backgrounds to account for the theoretical uncertainty of the
cross section calculation as well as the shape mismodeling.
\end{itemize}


\noindent We add all systematic uncertainties in quadrature and assign 
 20\% and 25\% relative uncertainties on the signal
acceptance and the Monte Carlo background predictions, respectively. The relative uncertainty of the 
less important backgrounds is 50\%.

