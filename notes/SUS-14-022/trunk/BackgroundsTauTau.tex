\section{Backgrounds for tauTau}
\label{sect:bkg}


\subsection{QCD multi-jet background estimation in tauTau channel}

In this section, data driven methods are applied to estimate the contribution of
   the main backgrounds in the signal region.


In QCD multi-jet events all tau candidates are misidentified as jets. Due to large cross
section and
the poor MC modeling of the tau misidentification rate from jets, the QCD multi-jet contribution in the SRs is estimated from data using the ABCD" method.

This method indeed relies on different distributions of QCD
in the four exclusive regions labelled as A, B, C (the control regions) and D (the signal region) are defined in a two-dimensional plane as a function of uncorrelated discriminating variables.
In this case the number of QCD events in signal region D can be calculated from the number of QCD events in the control region A multiplied in the ratio of the number of QCD events in the control region C to QCD events in control region B$(T=C/B)$.

The tau identification criterion (tau-id) and a kinematic variable chosen depending ($\Sigma M_{T}^{\tau}$ in Bin 1 and $MaxM_{T}^{\tau}$ in Bin2) 
on the SR are used as the two uncorrelated discriminating variables to define the regions A, B, C and D. The definitions of the control regions are summarized in table \ref{1QCDbg}.

\begin{table}
\begin{center}
\begin{tabular}{|c|c|c|c|}
\hline
Region&A& B & C
\\ \hline\hline
$\Sigma M_{T}^{\tau}>250$ Bin1 &$\Sigma M_{T}^{\tau} >$ 250&$\Sigma M_{T}^{\tau} <$250&$\Sigma M_{T}^{\tau} <$ 250\\
 &at least 2loose taus&at least 2 loose taus& \\
 &loose-loose loose-medium &loose-loose loose-medium &medium-medium \\
 &loose-tight&loose-tight&medium-tight tight-tight\\
 &No cut charge&No cut charge& Sum charge==0\\
% &misc.MinMetJetDphiPt40$>$1 is relaxed\\
\hline
$MaxM_{T}^{\tau}>200$ Bin2 &$Max M_{T}^{\tau} >$ 200&$Max M_{T}^{\tau} <$200&$Max M_{T}^{\tau} <$ 200\\
 &at least 2 loose taus&at least 2 loose taus& \\
 &loose-loose loose-medium &loose-loose loose-medium &medium-medium \\
 &loose-tight&loose-tight&medium-tight tight-tight\\ 
 &No cut charge&No cut charge& Sum charge==0\\

\hline
\end{tabular}
\caption{  The requirement on the  used to define the control regions A,B,C
.  }
\label{1QCDbg}
\end{center}
\end{table}

The number of QCD multi-jet events in the control regions is estimated from data after subtraction of other SM contributions estimated from MC simulation.

In order to increase the data statistics, the cut on the $misc.MinMetJetDphiPt40>1$ is relaxed.This cut was
introduced to suppress the QCD background events,now that we want to estimate QCD multi-jet background this cut is relaxed  .The only the ratio this efficiency should be
applied into account QCD in the control regions to estimate the number of QCD events in the signal region.

The fraction of QCD events with all selection cuts with respect to the QCD events with all selection cuts but the
$misc.MinMetJetDphiPt40>1$ are shown in Figure  


The distributions of the kinematic variables in the control regions A, B,C are shown in figure and the results of the ABCD method are summarized in table \ref{2QCDbg}.

\begin{table}
\begin{center}
\begin{tabular}{|l|c|c|c|c|c|c|c|}
\hline
 & Sample & RegionA & RegionB & RegionC & T=C/B & QCD in Signal Region(D)\\
\hline\hline
\multirow{7}{*}{$\Sigma M_{T}^{\tau}>250 $ Bin1}& Data& & & & \multirow{7}{*}{} & \multirow{7}{*}{}\\ \cline{2-5}

&Z+jets& & & & & \\\cline{2-5}

&W+jets& & & & & \\\cline{2-5}

&WW+jets& & & & & \\\cline{2-5}

&Top& & & & & \\\cline{2-5}
&QCD& & & & & \\\cline{2-5}
&Susy& & & & & \\\cline{2-5}
\hline\hline\hline
\multirow{7}{*}{$Max M_{T}^{\tau}>200$ Bin2}&Data  &  &  & \multirow{7}{*}{} & \multirow{7}{*}{} &  \\
\cline{2-5}


&Z+jets& & & & & \\\cline{2-5}

&W+jets& & & & & \\\cline{2-5}

&WW+jets& & & & & \\\cline{2-5}

&Top& & & & & \\\cline{2-5}
&QCD& & & & & \\\cline{2-5}
&Susy& & & & & \\\cline{2-5}
\hline\hline



\end{tabular}
\caption{ }
\label{2QCDbg}
\end{center}
\end{table}



















