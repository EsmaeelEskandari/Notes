\section{Information to test the new models}
\label{sect:model}
In the previous sections, a simplified SUSY model was used to optimize the selections and interpret the results. 
Here, the main efficiencies are reported versus the generated values, that can be used to examine the new models approximately in 
a MC generator-level study. The number of the remaining signal events and its uncertainty which can be evaluated by a generator-level study 
should be combined statistically with the results in Tab. \ref{tbl:yieldSysSummary} to find the upper limit on the number of the signal events
and decide if a model is excluded or still allowed in the shadow of the analysis presented in the paper.

In different channels, the generated taus are found. If it decays to leptons, 4-vector of the leptons and if it decays to hadrons, the difference between the 4-vector 
of the generated tau and the $\nu_{tau}$ are used as the generated particles to parametrize corresponding reconstructed particle. The latter 4-vector is referred to as the
visible \Tau. Table \ref{tbl:EffTauLep}
\begin{table}[!Hhtb]
\begin{center}
\begin{tabular}{|c|c|c|c|c|c|}
\hline\hline
generated \pt (\GeV)  & e for $e\Tau$ & $\mu$ for $\mu\Tau$  & \Tau for $\ell\Tau$    &  $\Tau^1$ for \tauTau & $\Tau^2$ for \tauTau\\
\hline\hline
0-10            &    0.1        &   0.1                &  0.1                   &       0.1           & 0.1\\\hline
10-20           &    0.1        &   0.1                &  0.1                   &       0.1           & 0.1\\\hline
20-30           &    0.1        &   0.1                &  0.1                   &       0.1           & 0.1\\\hline
30-40           &    0.1        &   0.1                &  0.1                   &       0.1           & 0.1\\\hline
40-60           &    0.1        &   0.1                &  0.1                   &       0.1           & 0.1\\\hline
60-80           &    0.1        &   0.1                &  0.1                   &       0.1           & 0.1\\\hline
80-120          &    0.1        &   0.1                &  0.1                   &       0.1           & 0.1\\\hline
120-160         &    0.1        &   0.1                &  0.1                   &       0.1           & 0.1\\\hline
160-200         &    0.1        &   0.1                &  0.1                   &       0.1           & 0.1\\\hline
$>$ 200         &    0.1        &   0.1                &  0.1                   &       0.1           & 0.1\\\hline
\hline
\end{tabular}
\caption{Efficiency to select a lepton or \Tau in different channels. $\Tau^1$ and $\Tau^2$ stand for leading and next-to-leading \Tau in the \tauTau channel.}
\label{tbl:EffTauLep}
\end{center}
\end{table}
shows the efficiency of selecting a lepton or \Tau for different channels versus the \pt of the generated lepton or visible \Tau. When \tauTau  or $\ell\Tau$ are selected, 
the negative of the sum of the 4-vector of the pair is used as the 4-vector of the generated missing particles. Its \pt is used as the generated \MET. Table \ref{tbl:EffMet}
\begin{table}[!Hhtb]
\begin{center}
\begin{tabular}{|c|c|c|}
\hline\hline
generated \MET (\GeV)  & $\ell\Tau$  &  \tauTau \\
\hline\hline
0-10            &    0.1        &   0.1   \\\hline
10-20           &    0.1        &   0.1   \\\hline
20-30            &    0.1        &   0.1   \\\hline
30-40            &    0.1        &   0.1   \\\hline
40-50            &    0.1        &   0.1   \\\hline
50-60            &    0.1        &   0.1   \\\hline
60-70            &    0.1        &   0.1   \\\hline
70-80            &    0.1        &   0.1   \\\hline
80-90            &    0.1        &   0.1   \\\hline
90-100            &    0.1        &   0.1   \\\hline
100-120            &    0.1        &   0.1   \\\hline
120-140            &    0.1        &   0.1   \\\hline
140-160            &    0.1        &   0.1   \\\hline
160-200            &    0.1        &   0.1   \\\hline
$>$ 200            &    0.1        &   0.1   \\\hline
\hline
\end{tabular}
\caption{Efficiency to pass the cut on \MET ($>$ 30 \GeV) in different channels versus the generated \MET.}
\label{tbl:EffMet}
\end{center}
\end{table}
shows the efficiency in different channels to pass the \MET $>$ 30 \GeV as a function of the generated \MET. The mass of the system of the selected pair is used to parametrize 
the efficiency to pass the cuts on the reconstructed invariant mass. Table \ref{tbl:EffMass}
\begin{table}[!Hhtb]
\begin{center}
\begin{tabular}{|c|c|c|}
\hline\hline
generated mass (\GeV)  & $\ell\Tau$  &  \tauTau \\
\hline\hline
0-5            &    0.1        &   0.1   \\\hline
5-10         &    0.1        &   0.1   \\\hline
10-15         &    0.1        &   0.1   \\\hline
15-20         &    0.1        &   0.1   \\\hline
20-25         &    0.1        &   0.1   \\\hline
25-30         &    0.1        &   0.1   \\\hline
30-35         &    0.1        &   0.1   \\\hline
35-40          &    0.1        &   0.1   \\\hline
40-45          &    0.1        &   0.1   \\\hline
45-50          &    0.1        &   0.1   \\\hline
50-55          &    0.1        &   0.1   \\\hline
55-60          &    0.1        &   0.1   \\\hline
60-65          &    0.1        &   0.1   \\\hline
65-70          &    0.1        &   0.1   \\\hline
70-75           &    0.1        &   0.1   \\\hline
75-80          &    0.1        &   0.1   \\\hline
80-85         &    0.1        &   0.1   \\\hline
85-90         &    0.1        &   0.1   \\\hline
90-95         &    0.1        &   0.1   \\\hline
95-100         &    0.1        &   0.1   \\\hline
$>$ 100         &    0.1        &   0.1   \\\hline
\hline
\end{tabular}
\caption{Efficiency to pass the cut on the invariant mass in different channels versus the generated mass.}
\label{tbl:EffMass}
\end{center}
\end{table}
shows the efficiency in different channels to pass the cut on the invariant mass ($>$ 15 \GeV) and ( $<$ 45 or $>$ 75 \GeV) for the $\ell\Tau$ channels 
and ( $<$ 55 or $>$ 85 \GeV) for the \tauTau channel. To parametrize the efficiency to pass the selection cuts on \mttwo, 4-vector of the generated lepton, \Tau and 
missing particles is used to calculate the generated \mttwo. The efficiency to pass the cut (\mttwo $>$ 90 \GeV) in $\ell\Tau$ and \tauTau \binone is shown in Tab. \ref{tbl:EffMT2}. 
\begin{table}[!Hhtb]
\begin{center}
\begin{tabular}{|c|c|c|}
\hline\hline
generated \mttwo (\GeV)  & $\ell\Tau$  &  \tauTau SR1 \\
\hline\hline
0-20    &    0.1        &   0.1   \\\hline
20-40 &    0.1        &   0.1   \\\hline
40-50 &    0.1        &   0.1   \\\hline
50-60 &    0.1        &   0.1   \\\hline
60-70 &    0.1        &   0.1   \\\hline
70-80 &    0.1        &   0.1   \\\hline
80-90 &    0.1        &   0.1   \\\hline
90-100 &    0.1        &   0.1   \\\hline
100-110 &    0.1        &   0.1   \\\hline
110-120 &    0.1        &   0.1   \\\hline
120-130 &    0.1        &   0.1   \\\hline
130-140 &    0.1        &   0.1   \\\hline
140-160 &    0.1        &   0.1   \\\hline
160-180 &    0.1        &   0.1   \\\hline
180-200 &    0.1        &   0.1   \\\hline
$>$ 200 &    0.1        &   0.1   \\\hline
\hline
\end{tabular}
\caption{Efficiency to pass the cut on \mttwo $>$ 90 \GeV in different channels versus the generated \mttwo.}
\label{tbl:EffMT2}
\end{center}
\end{table}
In the $\ell\Tau$ channels, to calculate the \tauMT, the 4-vector of the visible \Tau and missing particles are used. Table \ref{tbl:EffTauMT}
\begin{table}[!Hhtb]
\begin{center}
\begin{tabular}{|c|c|}
\hline\hline
generated \tauMT (\GeV)  & $\ell\Tau$ \\
\hline\hline
0-50   &   0.1   \\\hline
50-100  &   0.1   \\\hline
100-125  &   0.1   \\\hline
125-150  &   0.1   \\\hline
150-170  &   0.1   \\\hline
170-190  &   0.1   \\\hline
190-200  &   0.1   \\\hline
200-210  &   0.1   \\\hline
210-230  &   0.1   \\\hline
230-250  &   0.1   \\\hline
250-275  &   0.1   \\\hline
275-300  &   0.1   \\\hline
$>$ 300  &   0.1   \\\hline
\hline
\end{tabular}
\caption{Efficiency to pass the cut on \tauMT in $\ell\Tau$ channels versus the generated \tauMT.}
\label{tbl:EffTauMT}
\end{center}
\end{table}
shows the efficiency in the $\ell\Tau$ channels to pass the cut  \tauMT $>$ 200 \GeV versus the generated \tauMT.


In the \tauTau \bintwo, the reconstructed \mttwo is constrained between 40 and 90 GeV. Table \ref{tbl:EffMT2SR2}
\begin{table}[!Hhtb]
\begin{center}
\begin{tabular}{|c|c|}
\hline\hline
generated \mttwo (\GeV)  &  \tauTau \bintwo \\
\hline\hline
0-50   &   0.1   \\\hline
\hline
\end{tabular}
\caption{Efficiency to pass the cut on \mttwo in \tauTau \bintwo versus the generated \mttwo.}
\label{tbl:EffMT2SR2}
\end{center}
\end{table}
shows the efficiency in \tauTau \bintwo to pass the cut 40 $<$ \mttwo $<$ 90 \GeV versus the generated \mttwo. The last selection in this channel is
the cut on \SumMT which is calculated using the 4-vector of two \Tau and missing particles. Table \ref{tbl:EffSumMT} 
\begin{table}[!Hhtb]
\begin{center}
\begin{tabular}{|c|c|c|}
\hline\hline
generated \SumMT (\GeV)  &  \tauTau \bintwo\\
\hline\hline
0-50   &   0.1   \\\hline
\hline
\end{tabular}
\caption{Efficiency to pass the cut on the invariant mass in different channels versus the generated mass.}
\label{tbl:EffSumMT}
\end{center}
\end{table}
shows the efficiency in \tauTau \bintwo to pass the cut \SumMT $>$ 250 \GeV versus the generated \SumMT.

To use these efficiencies, one needs to multiply the values one after another and combine the final value with the values reported in Tab. \ref{tbl:yieldSysSummary} 
statistically, to decide if a signal point is not still excluded. In the generator level, only a pair of $\ell\Tau$ or \tauTau is selected and no other selection is applied.

To take into account the inefficiencies and misidentifications for charge reconstruction of the objects, b-tagging of the jets, identification of the extra leptons 
and the minimum angle between the jets and \MET in the transverse plane, the following extra factors need to be applied.



