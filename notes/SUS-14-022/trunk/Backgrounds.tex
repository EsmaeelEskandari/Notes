\section{Backgrounds}
\label{sect:bkgLepTau}
In different channels, the contribution of the events with a fake \Tau is usually estimated using the data. For the prompt \Tau's, we trust 
the MC, but the MC is validated in a signal like region. In continue different backgrounds and the the used method are listed.

\subsection{\texorpdfstring{QCD Background Estimation in $\tauTau$ Channel}{QCD Background Estimation in tau-tau Channel}}
Due to the large cross section of the QCD multijet events and lack of the statistics, there is a large statistical uncertainty on the 
yield of the QCD events from MC. On the other hand, these events contribute to the signal selection of the \tauTau channel, when two jets are 
fakely identified as \Tau's. The fake rate can be different between the data and MC, so a data driven method is developed to estimate the 
contribution of the QCD multijet events. 
An ABCD method is used when the two uncorrelated variables are the search variable (\mttwo in \binone and \SumMT in \bintwo) and the 
isolation of the \Tau's. In the signal region, loose \Tau's are excluded, but in the control region, only the pairs with at least a loose \Tau 
are selected. These pairs are requested to be same sign to suppress the signal contamination. To further increase the statistics 
in different regions, the cut on the minimum angle in the transverse plane between the \MET and the jets is removed. The efficiency of this
cut is multiplied to the final estimation. In the low \mttwo or \SumMT, the ratio of the signal like events over the same-sign loose pairs 
is found and verfied to be flat. 
%A horizental line is fitted to
This ratio is multiplied to the same-sign loose pairs in high \mttwo or \SumMT and corrected by the efficiency of the 
cut on the minimum angle between the \MET and the jets, to find the QCD contamination in the signal region. The method to extract the
ratio and the efficiency are varied to estimate the systematic uncertainty of the estimated value.



\subsection{\texorpdfstring{W+jets Background Estimation in $\tauTau$ Channel}{W+jets Background Estimation in tau-tau Channel}}
In \binone of the \tauTau channel, the number of remaining events for Wjets is zero, but the weight of the events is larger than one and 
it can introduce a large uncertainty on the yields of the backgrounds. To have a better estimation of the Wjets contribution in the final yields,
the yield before the last cut (\mttwo $>$ 90 \Gev) is multiplied by the efficiency of the cut. To find the efficiency, several cuts are relaxed 
to have a high statistics sample. The cut efficiency is found in the exclusive samples that either fail or pass each cut. 
A horizontal line is fitted to the measured values to extract the cut efficiency. The main source of the systematic uncertainty on the backgrounds 
is the \Tau energy scale. The energy of the \Tau's is scaled up and down by one standard deviation and all related variables are 
recalculated and the cut efficiency is meaured on the new samples. 
This variation due to the uncertainty of the \Tau energy scale is considered as the systematic uncertainty of the measured efficiency.
To validate the MC prediction for Wjets against the data, a Wjets enriched sample is made in \muTau channel, 
by rejecting the loosely tagged b-jets, relaxing the tau id from tight to loose and forcing the muon and \Tau to have the same sign. 
In the control sample, Wjets consist more than 90\% of the MC events. The normailzation and shape of the MC distribution  is consistent
with the data within the uncertainties.

\subsection{\texorpdfstring{DY Background Estimation in $\tauTau$ Channel}{DY Background Estimation in tau-tau Channel}}
The events containing a \Z boson can be an important background in different channels. To
evaluate this background, we trust the MC, but we validate the MC in a Z dominated area in
data.


\subsection{\texorpdfstring{Fake \Tau's in $\ell\Tau$ channels}{Fake Taus in Lepton-Tau Channels}}
A fake rate method is used to estimate the contribution of the fake \Tau's in $\ell\Tau$ channels. 
The idea is that when the loose signal selection is applied, the number of the loose $\hadtau$'s ($L$) is:
\begin{equation}
L = P + F
\end{equation}
where $P$ is the number of the  prompt $\hadtau$'s and $F$ is the number of the  fake $\hadtau$'s. If the selection is tightened, the number of the tight $\hadtau$'s (T) is
\begin{equation}
 T = pP + fF
\end{equation} 
$p$ ($f$) is the prompt (fake) rate, the probability that a loosely selected prompt (fake) $\hadtau$ can pass the  tight  selection. The loose category ($L$) can be divided to two parts, 
tight ($T$) and non-tight ($NT$), so one can write:
\begin{equation}
   F * (f - p) = ((1 - p) * L - NT)
\end{equation}
$f$ * $F$ is the contamination of the fake $\hadtau$'s in the signal region. 
The fake rate ($f$) is measured as the ratio of the tightly selected $\hadtau$'s to the loosely 
selected $\hadtau$'s in a sample which is dominated by the fake $\hadtau$'s. The fake rate is estimated in an environment which is exactly 
same as the signal selection, but the charges of the \Tau and lepton are similar instead of the opposite in the case of the signal selection.
The prompt rate ($p$) is measured in the MC DY events. The method is applied on MC events when only the Wjets events are selected and 
also on the full MC events with fake \Tau's. In both cases, the method closes properly and the estimated values lie within 
the uncertainties of the MC truth. The uncertainties include both statistical and systematical uncertainties which will be listed later in 
the paper. The uncertainties due to variation of the method to estimate the fake rate or prompt rate and their statistical uncertainties 
are found to be negligible.


