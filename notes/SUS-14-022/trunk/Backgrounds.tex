\section{Backgrounds}
\label{sect:bkgLepTau}
In different channels, the contribution of the events with a fake \Tau is usually estimated using the data. For the prompt \Tau's, we trust 
the MC, but the MC is validated in a signal like region. In continue different backgrounds and the the used method are listed.

\subsection{\texorpdfstring{Fake \Tau's in \ell\Tau channels}{Fake Taus in Lepton-Tau Channels}}
A fake rate method is used to estimate the contribution of the fake \Tau's in \ell\Tau channels. 
The idea is that when the loose signal selection is applied, the number of the loose $\hadtau$'s ($L$) is:
\begin{equation}
L = P + F
\end{equation}
where $P$ is the number of the  prompt $\hadtau$'s and $F$ is the number of the  fake $\hadtau$'s. If the selection is tightened, the number of the tight $\hadtau$'s (T) is
\begin{equation}
 T = pP + fF
\end{equation} 
$p$ ($f$) is the prompt (fake) rate, the probability that a loosely selected prompt (fake) $\hadtau$ can pass the  tight  selection. The loose category ($L$) can be divided to two parts, 
tight ($T$) and non-tight ($NT$), so one can write:
\begin{equation}
   F * (f - p) = ((1 - p) * L - NT)
\end{equation}
$f$ * $F$ is the contamination of the fake $\hadtau$'s in the signal region. 
The fake rate ($f$) is measured as the ratio of the tightly selected $\hadtau$'s to the loosely 
selected $\hadtau$'s in a sample which is dominated by the fake $\hadtau$'s. The fake rate is estimated in an environment which is exactly 
same as the signal selection, but the charges of the \Tau and lepton are similar instead of the opposite in the case of the signal selection.
The prompt rate ($p$) is measured in the MC DY events. The method is applied on MC events when only the Wjets events are selected and 
also on the full MC events with fake \Tau's. In both cases, the method closes properly. 
