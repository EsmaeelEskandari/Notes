\section{Monte Carlo samplesdghjghjgh}
\label{sect:MCSamples}
The events of $\cPZ$+jets, \wjets, $\cPqt\cPaqt$, and di-boson, backgrounds to this search, are generated using the \MADGRAPH 5.1~\cite{Alwall:2011uj} generator. 
Single-top-quark and Higgs boson background events are generated by {\POWHEG} 1.0~\cite{Nason:2004rx,Frixione:2007vw,Alioli:2009je,Alioli:2010xd}.
In the following figures and tables, the events containing at least a top quark and $\cPZ$ are refered to as ``Top'' and ``ZX'', respectively. 
Different Higgs boson productions, gluon fusion, vector boson fusion and associated production of Higgs with $\cPZ$ or $\PW$ or $\cPqt\cPaqt$ are considered and referred to as ``Higgs''. To generate the Monte Carlo events the masses of the top quark and Higgs boson are 172.5\GeV and 125\GeV, respectively.
For parton shower and fragmentation, all generators are interfaced with \PYTHIA 6.4~\cite{Sjostrand:2006za}.
\PYTHIA is also used to generate signal events (chargino pair-production). To improve the modeling of $\Pgt$ decays, we use the  \TAUOLA~\cite{Davidson:2010rw} package.
In the dataset considered in this paper,
there were on average 21 proton-proton interactions (``pileup'') in each bunch-crossing.
Consequently, additional interactions are generated with \PYTHIA and superimposed on Monte Carlo events in a manner consistent with the
luminosity profile of the dataset.
The detector response in the Monte Carlo background event samples is modelled by a
detailed simulation
of the CMS detector based on {\GEANTfour}~\cite{Agostinelli:2002hh}.  
On the other hand, in order to reduce  computational requirements, signal events 
are processed by the CMS fast simulation \cite{Abdullin:2011zz} instead of {\GEANTfour}. 
All simulated events are reconstructed with the same algorithms as collision data.
The SM backgrounds are normalized using the most accurate calculations of the cross sections available,
generally with next-to-leading-order (NLO) or next-to-next-to-leading-order accuracy~\cite{Campbell:2012uf,Campbell:2011bn,xsec_WZ,Czakon:2013goa}.
The \textsc{Resummino}~\cite{Fuks:2012qx,Fuks:2013vua,Fuks:2013lya} calculations at NLO+NLL are used to calculate the signal cross sections, where 
NLL refers to next-to-leading-logarithmic precision.

