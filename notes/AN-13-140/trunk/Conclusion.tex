\section{Conclusion}
\label{sect:conclusion}
A hadronic search for direct Stop production is presented using the \mttwo variable. Data driven methods are used to estimate the main
backgrounds. It is shown that the methods close properly on MC. Since this analysis uses a multijet trigger and \mttwo does not 
depend explicitly on \met, this analysis can be complementary to the common cut and count search for Stop which uses \met trigger. 
It is shown that in the regions with low mass difference between $m_{\tilde{t}}$ and 
$m_{LSP}$ this analysis can have a comparable reach. There is a plan to optimize the cuts to improve the reach in this region and exclude the
masses which are not reachable by the other analyses.

\section{Outlook of the Analysis}
In next step, the analysis will be updated to the full dataset of 2012 which sums up to about 20 \invfb of data. 
There is an idea to look at the "Parked data" also. Thanks to the looser triggers in the "Parked data", 
it is expected to have a better signal efficiency when the analysis applied on this data. 
Re-optimizing the analysis with these looser triggers is foreseen for next steps to improve the reach even further.

\section{Acknowledgements}
This analysis benefits highly from the computing resources and codes developed in ETHZurich. 
We appreciate their help and generosity. The method used for the top reconstruction was firstly introduced and implemented by L. Pape. 
We thank him for providing us the code. The authors would like to thank the conveners of the SUSY-TBT working group, Rick Cavanaugh 
and Juan Alcaraz Maestre, for their help and support. The authors would like to thank the management and staff of the school of particles 
and accelerators of IPM, especially Prof. Arfaei for their help and support. Thanks to all of the members of
the CMS collaboration for their outstanding results discussed partly here.
