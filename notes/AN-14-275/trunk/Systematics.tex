\section{Systematic Uncertainties}
\label{sect:sys}
\subsection{Signal Systematic Uncertainties}
\subsubsection{ISR}
The jet activity of our signal is coming from Initial State Radition(ISR) and due to generating our signal with PYTHIA generator which can not simulate ISR properly, taking into account only $2 \rightarrow 2$ scattering processes, therefore ISR jet momenta spectra as well as the multiplicity cannot be Trusted. Because of this issue it is possible to generate lightc-flavor jet which can be tagged wrongly as a b-jet and by applying a b-veto cut we miss that signal event. On the other side MADGRAPH generator, which is a matrix element generator provids a better description of ISR jets and because it can simulate $2 \rightarrow 4$ interactions. 
We take advantage of this point by choosing a signal-like MADGRAPH generated MC sample like $WW$ which mimics the signal from the generation point of view and comparing it with the signal point whose M(\chione) = 100 GeV and M(\nuetraliono) = 0 GeV and this point resembles WW with respect to $\sqrt {\hat{s}}$. It is followd the below recipe, originating from $fixME$.
By applying and relaxing the b-veto cut it is obtained the yields of WW and signal and by the uncertainty is Unc1:

%\begin{linenomath}
%\begin{equation}
%\label{eq:suWW}
%I_{e/ \mu}=\Sigma P_{T}^{charged }(\Delta z <2mm)+max(P_{T}^{photon}+P_{T}^{h0 }-\Delta \beta,0),

\begin{align}
WW\_ratio &= \frac{WW\_bveto}{WW\_relaxed} = \frac{88968.51}{93058.33} = 0.953\\ \nonumber
susy\_ratio &= \frac{susy\_bveto}{susy\_relaxed} = \frac{2041.88}{2206.07} = 0.922 \\ \nonumber
Unc1 &= \frac{susy\_ratio - WW\_ratio}{WW\_ratio} = 3 \% 
\end{align}

%\end{equation}
%\end{linenomath}

Then due to not having a WW enriched region in data we use the previous WW sample multiplied by 0 b-tag SF as data and we redo the above calculation for data ($WW \times 0b\_tag SF$) means applying and relaxing b-tagging and we have Unc2:
%\begin{linenomath}
\begin{align}
%\label{eq:suWW}
%I\_{e/ \mu}=\Sigma P\_{T}^{charged }(\Delta z <2mm)+max(P\_{T}^{photon}+P\_{T}^{h0 }\_\Delta \beta,0),
data\_ratio &= \frac{data\_bveto}{data\_relaxed} = \frac{80282.30}{84226.93} = 0.956 \\ \nonumber
Unc2 &= \frac{data\_ratio - WW\_ratio}{WW\_ratio} = 0.03 \%
\end{align}
%\end{linenomath}

and finally the ISR systemtic uncertaintiy due to B-mistagged jet veto is 3 \%.

\subsubsection{Systematic uncertainty due to $min\delta \phi (\met , Jets)$}

The other source of systematic uncertainty due to existence of ISR in signal events is using of $min\delta \phi (\met , Jets)$ ($ > 1$) variable against QCD events. We use the same method as above to obtain the systematic uncertainty of applying this variable. But this time we use Higgs to $\tau \tau$ sample which mimics our signal like WW and we chose it because Higgs sample has the most similar cumulative distribution function,i.e. the efficiency of this cut, to our signal that is shown in. %Fig.~\ref{}.     

%% \begin{align}

%% Higgs\_ratio &= \frac{Higgs\_bveto}{Higgs\_relaxed} = \frac{88968.51}{93058.33} = 0.953\\ \nonumber
%% susy\_ratio &= \frac{susy\_bveto}{susy\_relaxed} = \frac{2041.88}{2206.07} = 0.922 \\ \nonumber
%% Unc1 &= \frac{susy\_ratio - Higgs\_ratio}{Higgs\_ratio} = 2\% 

%% \end{align}

and finally the ISR systemtic uncertaintiy due to B-mistagged jet veto is 2 \%.

%% \begin{table}[!h]
%% \begin{center}
%% \begin{tabular}{|c|c|c|c|c|c|c|}
%% \hline
%%  Higgs\_ratio & susy\_ratio & data\_ratio & Unc1 & Unc2 & total uncertainty \\\hline 
%%            &             &             &      &      &                    \\     \hline   

%% \end{tabular}
%% \end{center}
%% \caption{a
%% }
%% \label{Tab.susyHiggs}
%% \end{table}     

%%            & WW\_ratio & susy\_ratio & data\_ratio & Unc1 & Unc2 & total uncertainty \\\hline 
%% $e \tau_{had} channel$                  &           &             &             &      &      &                    \\     \hline   
%% $\mu \tau_{had} channel$                  &           &             &             &      &      &                    \\     \hline   
%% $\tau_{had} \tau_{had} channel bin1$                  &           &             &             &      &      &                    \\        \hline
%% $\tau_{had} \tau_{had} channel bin2$                  &           &             &             &      &      &                    \\        \hline
\subsubsection{Tau Energy Scale}
Tau pt is varieid by 5\% up and down followd by Tau POG~\cite{TauPOG} recommendation ,and then it is taken into account the effect of this variation on \met and \pT relateds variables and we obtaind the up and down changes of different variables on final yields in different channels which is reported in table.
   

\subsubsection{Pile-up}
