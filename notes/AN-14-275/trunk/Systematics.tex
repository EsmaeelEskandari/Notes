\section{Systematic Uncertainties}
\label{sect:sys}
\subsection{Signal Systematic Uncertainties}
\subsubsection{Luminosity}
The uncertainty on the luminosity  is $2.6\%$ for $2012$ data ~\cite{LUMI}.

\subsubsection{ISR}
The jet activity of our signal is coming from Initial State Radiation(ISR). Our signal is generated with \PYTHIA event generator which can not simulate ISR properly and this is because of that it can consider only $2 \rightarrow 2$ scattering processes. Therefore ISR jet momenta spectra as well as the multiplicity cannot be trusted. Because of this issue it is possible to generate light jets which can be tagged wrongly as a b-jet and by applying a b-veto cut we miss that signal event. On the other side \MADGRAPH generator, which is a matrix element generator provides a better description of ISR jets which it can simulate $2 \rightarrow 4$ interactions. 
We take advantage of this point by choosing a signal-like \MADGRAPH generated MC sample which mimics the signal from the generation point of view and comparing it with the signal point whose $m(\chione) = 100\,\GeV$ and $m(\PSGczDo) = 0\,\GeV$. This point resembles WW with respect to $\sqrt {\hat{s}}$.
By applying and relaxing the b-veto cut it is obtained the yields of WW and signal:

%\begin{linenomath}
%\begin{equation}
%\label{eq:suWW}
%I_{e/ \mu}=\Sigma P_{T}^{charged }(\Delta z <2mm)+max(P_{T}^{photon}+P_{T}^{h0 }-\Delta \beta,0),

\begin{align}
WW\_ratio &= \frac{WW\_bveto}{WW\_relaxed} = \frac{88968.51}{93058.33} = 0.953\\ \nonumber
susy\_ratio &= \frac{susy\_bveto}{susy\_relaxed} = \frac{2041.88}{2206.07} = 0.922 \\ \nonumber
A &= \frac{susy\_ratio - WW\_ratio}{WW\_ratio} = 3 \% \nonumber
\end{align}

%\end{equation}
%\end{linenomath}

Then due to not having a WW enriched region in data we use the previous WW sample multiplied by 0 b-tag SF as data and we redo the above calculation for data (WW $\times {\rm 0b\_tag SF}$) means applying and relaxing b-tagging and we have B:
%\begin{linenomath}
\begin{align}
%\label{eq:suWW}
%I\_{e/ \mu}=\Sigma P\_{T}^{charged }(\Delta z <2mm)+max(P\_{T}^{photon}+P\_{T}^{h0 }\_\Delta \beta,0),
data\_ratio &= \frac{data\_bveto}{data\_relaxed} = \frac{80282.30}{84226.93} = 0.956 \\ \nonumber
B &= \frac{data\_ratio - WW\_ratio}{WW\_ratio} = 0.003 \% \nonumber
\end{align}
%\end{linenomath}

and finally the ISR systematic uncertainty due to B-mistagged jet veto is 3 \%.

\subsubsection{\texorpdfstring{Systematic uncertainty due to $\mindphifour$}{Systematic uncertainty due to minDeltaPhi4}}
The other source of systematic uncertainty due to existence of ISR in signal events is using of $\mindphifour$ ($ > 1$) variable against QCD events. We use the same method as above to obtain the systematic uncertainty of applying this variable. But this time we use Higgs to $\tau \tau$ sample which mimics our signal like WW.
%and we chose it because Higgs sample has the most similar cumulative distribution function,i.e. the efficiency of this cut, to our signal that is shown in. %Fig.~\ref{}.     

\begin{align}
Higgs\_ratio &= \frac{Higgs\_bveto}{Higgs\_relaxed} = \frac{89288.14}{180562.56} = 0.491 \\ \nonumber
susy\_ratio &= \frac{susy\_bveto}{susy\_relaxed} = \frac{1020.33}{2206.07} = 0.466 \\ \nonumber
A &= \frac{susy\_ratio - Higgs\_ratio}{Higgs\_ratio} = 3 \% 
\end{align}


\subsubsection{\hadtau Energy Scale}

%The electron energy scale is varied by $1\%$ ($2.5\%$) for electrons reconstructed in the barrel (endcap) region of the ECAL.
%The uncertainty on muon momenta is assumed to be negligible. 
The energy scale of hadronically decaying taus is varied by $3\%$, following the recommendation of the Tau POG~\cite{TauPOG} and it is taken into account the effect of this variation on \MET and \pt related variables and we obtained the up and down changes of different variables on final yields in different channels which is reported in table.



%% \begin{table}[!h]
%% \tiny{
%% \begin{center}
%% \begin{tabular}{|c|c|c|c|c|c|c|c|c|}
%% \hline
%%                              & QCD & ZX    & W  & WW   & Top    & All MC & Susy & Data \\\hline 
%% $e\hadtau$ channel           & $0.0 ^{+0.0} _{-0.0} $ & $0.38 ^{+0.05} _{-0.03}  $    &  $1.2 ^{+0.0} _{-0.0} $      &  $0.05 ^{+0.0} _{-0.0} $   &$0.02 ^{+0.12} _{-0.0} $           & $1.74 ^{+0.13} _{-0.03} $       & $3.47 ^{+0.33} _{-0.0} $ & $3.0 ^{+0.0} _{-1.0}$    \\\hline   

%% $\mu\hadtau$ channel &  $0.0 ^{+0.0} _{-0.0} $     &  $0.28 ^{+0.11} _{-0.0} $      &  $0.79 ^{+0.0} _{-0.32} $  & $0.34 ^{+0.07} _{-0.05} $        &  $0.0 ^{+0.04} _{-0.06} $   &    $1.4 ^{+0.22} _{-0.34} $      &  $2.41 ^{+0.17} _{-0.02} $      & $5.0 ^{+1.0} _{-2.0} $     \\\hline  

%% $\tauTau$ channel bin1 &  $0.0 ^{+0.0} _{-0.0}$   &    $0.56 ^{+0.7} _{-0.09}$    &  $0.0 ^{+0.0} _{-0.0}$      &  $0.02 ^{+0.0} _{-0.02}$        &   $0.0 ^{+0.0} _{-0.0}$        &    $0.58 ^{+0.7} _{-0.11}$     & $4.1^{+0.24} _{-0.26} $    & $0.0 ^{+0.0} _{-0.0}$\\\hline

%% $\tauTau$ channel bin2 &  $0.0 ^{+0.0} _{-0.0}$   &     $0.81 ^{+0.39} _{-0.7}$     &    $0.43 ^{+0.0} _{-0.0}$     &     $0.15 ^{+0.1} _{-0.02}$     &   $0.53 ^{+0.0} _{-0.23}$   &      $1.91 ^{+0.28} _{-1.94}$     &     $3.13 ^{+0.12} _{-0.27}$   &  $0.0 ^{+0.0} _{-0.0}$    \\\hline
%% \end{tabular} 
%% \end{center}
%% \caption{Tau energy scale systematic effect on MC in different channels}
%% \label{Tab.susyHiggs}

%% }
%% \end{table}     

%% \begin{table}[!h]
%% \tiny{
%% \begin{center}
%% \begin{tabular}{|c|c|c|c|c|c|c|c|c|}
%% \hline
%%                              & QCD & ZX    & W  & WW   & Top    & All MC & Susy & Data \\\hline 
%% $e\hadtau$ channel           & $0.0 ^{+0.0 \%} _{-0.0 \%} $ & $0.38 ^{+13 \%} _{-5 \%}  $    &  $1.2 ^{+0.0 \%} _{-0.0 \%} $      &  $0.05 ^{+0.0 \%} _{-0.0 \%} $   &$0.02 ^{+600.0 \%} _{-0.0 \%} $           & $1.74 ^{+8 \%} _{-2 \%} $       & $3.47 ^{+9 \%} _{-0.0 \%} $ & $3.0 ^{+0.0 \%} _{-33 \%}$    \\\hline   

%% $\mu\hadtau$ channel &  $0.0 ^{+0.0 \%} _{-0.0 \%} $     &  $0.28 ^{+40 \%} _{-0.0 \%} $      &  $0.79 ^{+0.0 \%} _{-40 \%} $  & $0.34 ^{+20 \%} _{-15 \%} $        &  $0.0 ^{+0.0 \%} _{-0.0 \%} $   &    $1.4 ^{+ 16 \%} _{- 24 \%} $      &  $2.41 ^{+7 \%} _{-0.0 \%} $      & $5.0 ^{+20 \%} _{-40 \%} $     \\\hline  

%% $\tauTau$ channel bin1 &  $0.0 ^{+0.0 \%} _{-0.0 \%}$   &    $0.56 ^{+12 \%} _{-16 \%}$    &  $0.0 ^{+0.0 \%} _{-0.0 \%}$      &  $0.02 ^{+0.0 \%} _{-100 \%}$        &   $0.0 ^{+0.0 \%} _{-0.0 \%}$        &    $0.58 ^{+12 \%} _{-19 \%}$     & $4.1^{+5 \%} _{-5 \%} $    & $0.0 ^{+0.0 \%} _{-0.0 \%}$\\\hline

%% $\tauTau$ channel bin2 &  $0.0 ^{+0.0 \%} _{-0.0 \%}$   &     $0.81 ^{+48 \%} _{-9 \%}$     &    $0.43 ^{+0.0 \%} _{-0.0 \%}$     &     $0.15 ^{+66 \%} _{-13 \%}$     &   $0.53 ^{+0.0 \%} _{-43 \%}$   &      $1.91 ^{+15 \%} _{-100 \%}$     &     $3.13 ^{+4 \%} _{-8 \%}$   &  $0.0 ^{+0.0 \%} _{-0.0 \%}$    \\\hline
%% \end{tabular} 
%% \end{center}
%% \caption{Tau energy scale systematic effect on MC in different channels}
%% \label{Tab.susyHiggs}
%% }
%% \end{table}     


\begin{table}[!h]
\tiny{
%\begin{center}
\begin{tabular}{|c|c|c|c|c|c|c|c|c|}
\hline
                             & QCD & ZX    & W  & WW   & Top    & All MC & Susy & Data \\\hline 
$e\hadtau$ channel           & $0.0\pm0.0^{+0.0}_{-0.0} $ & $0.38\pm0.06^{+0.13}_{-0.08}$ & $1.29\pm0.62^{+0.0}_{-0.0} $  & $0.05\pm0.04^{+0.0\%}_{-0.0\%} $ &$0.02\pm0.02^{+78.0\%} _{-0.0\%}$  & $1.74\pm0.63^{+8\%}_{-2\%}$ & $3.47 ^{+9 \%} _{-0.0 \%} $ & $3.0\pm1.73 ^{+0.0 \%} _{-33 \%}$    \\\hline   

$\mu\hadtau$ channel &  $0.0 ^{+0.0 \%} _{-0.0 \%} $     &  $0.28 \pm 0.05 ^{0.14} _{-0.06} $      &  $0.79 \pm 0.47^{0.8} _{0.35} $  & $0.34 \pm 0.14 ^{0.37} _{0.24} $        &  $0.0\pm0.0 ^{+0.67} _{-0.06} $   &    $1.4 \pm 0.49 ^{} _{} $      &  $2.26 \pm 0.35^{+0.17} _{-0.19} $      & $5.0 ^{} _{} $     \\\hline  

$\tauTau$ channel bin1 &  $0.0\pm 0.0 ^{+0.0 \%} _{-0.0 \%}$   &    $0.56 \pm 0.07 ^{+0.7} _{-0.09}$    &  $0.0 \pm 0.0 ^{+0.0} _{-0.0}$      &  $0.02 \pm 0.02 ^{+0.0} _{0.02}$        &   $0.0 \pm 0.0 ^{+0.0 \%} _{-0.0 \%}$        &    $0.58 \pm 0.07 ^{} _{}$     & $4.1 \pm 0.28^{} _{} $    & $1.0 \pm1.0 ^{+0.0 \%} _{-0.0 \%}$\\\hline

$\tauTau$ channel bin2 &  $0.0 \pm 0.0 ^{+0.0 \%} _{-0.0 \%}$   &     $0.81 \pm 0.56 ^{+0.39} _{-0.7}$     &    $0.43 \pm 0.4 ^{+0.0 \%} _{-0.0 \%}$     &     $0.15 \pm 0.07 ^{0.0} _{-0.02}$     &   $0.53 \pm 0.53 ^{+0.0} _{0.0}$   &      $1.91 \pm 0.87 ^ {} _{}$     &     $3.13 \pm 0.24 ^{} _{}$   &  $2.0 \pm 1.41 ^{} _{}$    \\\hline
\end{tabular} 
%\end{center}
\caption{Tau energy scale systematic effect on MC in different channels}
\label{Tab.susyHiggs}
}
\end{table}     


\subsubsection{Lepton trigger, identification, isolation efficiency}

  The uncertainties on electron and muon trigger, identification and isolation efficiencies are $2\%$ for electrons ~\cite{CMS_AN_2013-171} as well as muons of all $P_{T}$ and $\eta$.  The uncertainty on the $\hadtau$ identification efficiency amounts to $6\%$ ~\cite{CMS_AN_2013-171}
  The uncertainty on the efficiency of the hadronic tau leg of the $e\hadtau$ and $\mu\hadtau$ ($\tauTau$) trigger
  amounts to $3.0\%$ ($4.5\%$ per leg).

\subsubsection{PDF}
The effect of PDF on cross section is considered using $CTEQ66$ and $MSTW2008nlo90cl$ for PDF and it is shown for some generic susy points in Table.~\ref{Tab.PDF}.
\begin{table}[!h]
\begin{center}
\begin{tabular}{|c|c|c|}
\hline
                                    &$\sigma (fb) \_ CTEQ66$          & $\frac{\sigma \_ CTEQ66 - \sigma \_ MSTW2008}{\sigma \_ CTEQ66}$  \\\hline 
m(\chione) = 100 GeV                &$5823.40^{+0.0 \% + 3.4 \%}_{-0.6 \% - 3.2 \%}$         & 3 \%         \\\hline   
m(\chione) = 200 GeV                &$379.24^{+0.4 \% + 4.5 \%}_{-0.4 \% - 4.4 \%}$          & 6 \%        \\\hline  
m(\chione) = 300 GeV                &$67.51^{+0.2 \% + 5.9 \%}_{-0.2 \% - 5.1 \%}$           & 7 \%        \\\hline
m(\chione) = 400 GeV                &$17.51.40^{+0.0 \% + 6.8 \%}_{-0.3 \% - 6.3 \%}$        & 8 \%        \\\hline
m(\chione) = 500 GeV                &$5.53^{+0.0 \% + 8.1 \%}_{-0.9 \% - 7.0 \%}$            & 12 \%        \\\hline
\end{tabular} 
\end{center}
\caption{Xsection systematic uncertainty due to different PDF
}
\label{Tab.PDF}
\end{table}     
\subsubsection{Pile-up}

We varied minimum bias cross section $5 \%$ up and down following ~\cite{PU_SYS}  , and it causes  $7 \%$ systematic for all channels.    