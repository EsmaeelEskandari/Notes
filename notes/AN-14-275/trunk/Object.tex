\section{Physics Object Definition}
\label{sect:objdef}
To define the physics objects of the analysis, we follow the recommendations and selections of another CMS analysis which search for Higgs bosons decaying to $\tau$ pairs \cite{CMS_AN_2013-188}. Due to the very similar final states, it is well motivated to avoid duplication of the efforts to define and optimize the object selections. For the completeness of the note, the object selections are reviewed shortly here. More detail can be found in \cite{CMS_AN_2013-188} and \cite{HiggsTauTautwiki}

\subsection{Electron}
Electron identification is based on a MVA (Boosted Decision Tree) method ~\cite{Hocker:2007ht} which also cleans the colletion of electron from jet\,$\rightarrow e$  fakes. The training process takes advantage of two bins of $\pt$ and three bins of $\eta$, running over a sample of $Z \rightarrow ee$  events selected in data. Those oppositely charged electrons pair closest to Z-boson peak mass is taken as "signal" and the other electron candidates as "background". The training input variables are described in ~\cite{CMS_AN_2013-188}. Different discriminators are trained on electrons passing the criteria of single electron trigger and also on non triggered electrons. In this analysis the latter case is chosen.   
Depending on the BDT output, a loose and tight working point of \textit{ElIDMVANoTrig} algorithm can be defined. The citeria in different bins of $\pt$ and $\eta$ are given in Table ~\ref{Tab.electronMVAIDwp}.

\begin{table}[!h]
\begin{center}
\begin{tabular}{|l|l|c|c|}
\hline\hline
\multicolumn{2}{|c|}{Kinematic range}                & Loose & Tight \\
\hline\hline
\multirow{3}{*}{$\pt < 20$~\GeV} & $\vert \eta \vert < 0.8$         & 0.925 & 0.925 \\
                  & $0.8 < \vert \eta \vert < 1.479$ & 0.915 & 0.915 \\
                  & $\vert \eta \vert > 1.479$       & 0.965 & 0.965 \\\hline
\multirow{3}{*}{$\pt > 20$~\GeV} & $\vert \eta \vert < 0.8$         & 0.905 & 0.925 \\
                  & $0.8 < \vert \eta \vert < 1.479$ & 0.955 & 0.975 \\
                  & $\vert \eta \vert > 1.479$       & 0.975 & 0.985 \\
\hline\hline
\end{tabular}
\end{center}
\caption{
  Discriminator thresholds for the Loose and Tight MVA electron identification working--points respectively.
}
\label{Tab.electronMVAIDwp}
\end{table}     

To reject those electrons which is coming from photon conversions, it is required that the electron touches all layers of Pixel detector. Furthermore if there is an oppositely charged track near electron which is fitted to a common vertex inside of tracker volume that electron is rejected.  
\subsection{Muon}
Muons are required to be reconstructed by the Tracker or the Global muon reconstruction algorithm and to be identified as 
muons by tight particle flow algorithm.

The particle flow algorithm identifies  muons by applying several criteria.

$\bullet$ The number of pixel hit(s) associated to muon track\,$\geq $\,1

$\bullet$ Number of tracker layers with hits should be $\geq $\,6

$\bullet$ The number of hits in muon system $\geq $\,1

$\bullet$ Tracker track matched with at least one muon segment (in any station)

$\bullet$ $\chi ^{2}/{\rm NDF} $ for the global track fit\,$< $\,10 ,

$\bullet$ Impact parameter constrains between the muon track and the selected primary vertex 
 $d_{z} < $\,0.5 cm and $d_{0} <$\,0.2 cm.

\subsection{Muon and electron isolation}

In order to reduce the background contributions from QCD multi–jet events, electrons and
muons are required to be isolated. The isolation is computed as the $\pt$ sum of charged particles (including charged hadrons, 
electrons and muons), neutral hadrons plus photons reconstructed by the PF algorithm within a cone of size
$\Delta R_{\rm iso}$ = 0.4 around the direction of the muon respectively electron. 

In the innermost region
(``veto cone'') neutral
hadrons and photons  are excluded from the computation of the isolation $\pt$ sum in order to prevent energy deposits in the electromagnetic and hadronic calorimeters. Charged particles close to the direction of electrons  are excluded too in order to avoid tracks due to conversions of photons emitted by Bremsstrahlung processes to spoil the isolation.

For the lepton isolation the photon and neutral hadron candidates are required to have a transverse energy of $\pt>0.5\,\GeV$. This reduces pile-up effects. The correction of pile-up to isolation is done by applying $\Delta\beta$ corrections.
\begin{linenomath}
\begin{equation}
\label{eq:muonisolation1}
I_{e/ \mu}\,=\,\Sigma\, \pt^{\rm charged }(\Delta z <2\,mm)+max(\pt^{\rm photon}+\pt^{h0 }-\Delta \beta,0),
\end{equation}
\end{linenomath}
The $\Delta\beta$ corrections are computed by summing the transverse momenta of charged particles
that have longitudinal impact parameters $\Delta z > 2$ mm with respect to the lepton production
vertex and scaling the sum by a factor 0.5:
\begin{linenomath}
\begin{equation}
\label{eq:muonisolation2}
\Delta\beta\,=\,0.5\, \Sigma \pt^{\rm charged }(\Delta z > 2mm).
\end{equation}
\end{linenomath}

\subsection{\texorpdfstring{Hadronic $\tau$}{Hadronic tau}} 
\label{sec:hadTau}
The "Hadron plus Strip" (HPS) algorithm~\cite{2012JInst...7.1001C} is employed to reconstruct the hadronic decays of $\tau$ leptons with the jet constituents as input. The algorithm is seeded by PF jets reconstructed using the anti-$k_{\rm T}$ procedure with a distance parameter of $R=0.5$. To discriminate against quark and gluon jets, jets with extra particles not compatible with the hadronic $\tau$ decays are rejected. Additional criteria are used to suppress the contributions from electrons and muons in the hadronic $\tau$ ($\hadtau$) collection.

Based on the expected particles in the final state of the hadronic $\tau$ decay, different combinations of charged hadrons and $\pi^0$ candidates are considered. The $\pi^0 (\to\gamma\gamma)$ candidates are reconstructed using PF photons with $\pt>2.5\,\GeV$, falling into $\eta-\phi$ "strips" with specific size~\cite{CMS_AN_2013-171}. While a \textbf{single charged hadron} is an indication of $\tau^{\pm}\to\pi^{\pm}\nu_{\tau}$, a combination of \textbf{three charged hadrons} fulfilling a set of requirements are considered as the process of $\tau^{\pm}\to a_1^{\pm}\nu_{\tau}\to\pi^{\pm}\pi^{\mp}\pi^{\pm}\nu_{\tau}$. \textbf{One charged hadron plus two Strips} with a dedicated selection is assigned to $\tau^{\pm}\to a_1^{\pm}\nu_{\tau}\to\pi^{\pm}\pi^{0}\pi^{0}\nu_{\tau}$ decays. Finally the signature of  $\tau^{\pm}\to\rho^{\pm}\nu_{\tau}\to\pi^{\pm}\pi^{0}\nu_{\tau}$ decay is searched for in \textbf{one charged hadron plus one Strip} combinations. All charged hadrons and strips in the $\tau$ decay mode reconstruction must be within a narrow cone around the jet axis where the cone size depends on the $\tau$ jet $\pt$. Details on specific selections for each category as well as the $\pt$ dependence of the cone size can be found in Ref.~\cite{CMS_AN_2013-171}

The isolation variable is defined as the sum of the $\pt$ of charged hadrons and the $\et$ of photons within a cone of $\Delta R = 0.5$ around the $\hadtau$ axis. Particles used to reconstruct the $\hadtau$ candidate are excluded. The contribution of pile-up to the $\Tau$ isolation is accounted for by applying the so-called $\Delta\beta$ corrections. The selection criteria for charged hadrons in the isolation sum together with the pile-up subtraction procedure can be found in Ref.~\cite{CMS_AN_2013-171}. The Loose, Medium and Tight working points of {\it CombinedIsoDBSumPtCorr3Hits} algorithm correspond to isolation variables less than 2.0, 1.0 and 0.8\,$\GeV$, respectively.

The $\hadtau$ candidates are vetoed if signals exist in the muon system close to the $\hadtau$ direction. Loose, Medium and Tight working points are provided, corresponding to different $\hadtau$ identification efficiencies and $\mu\to\hadtau$ fake rates~\cite{CMS_AN_2013-171}. To discriminate against electrons i.e., rejecting $e\to\hadtau$ fakes, a multivariate discriminator is trained for which the Loose, Medium, Tight and very-Tight working points are defined based on the outputs of 16 different BDT's. Each BDT is associated to a category of $\hadtau$ candidates where the classification of $\hadtau$'s is performed according to their kinematics and decay mode as well as their closeness to a GSF electron. The working points are optimized to achieve the lowest $e\to\tau$ rate possible with a given $\hadtau$ identification efficiency~\cite{CMS_AN_2012-417}.

%Tau leptons from Higgs boson decays are expected to be isolated in the detector, while leptons from heavy-flavor (c and b) decays and decays in flight are expected to be found inside jets. A measure of isolation is used to discriminate the signal from the QCD multijet background, based on the charged hadrons, photons, and neutral hadrons falling within a cone around the lepton momentum direction.
%Electron, muon, and tau lepton isolation are estimated as
%\begin{equation}\begin{aligned}
%I_{\Pe,\Pgm} &=  \sum_{\rm charged}  \pt + \text{max}\left( 0, \sum_{\rm neutral}  \pt
%                                        +  \sum_{\gamma} {\pt} - 0.5 \sum_{\rm charged, pileup} \pt  \right ), \\
%I_{\Tau} &=  \sum_{\rm charged}  \pt + \text{max}\left( 0, \sum_{\gamma} {\pt} - 0.46 \sum_{\rm charged, pileup} \pt  \right ),
%\label{eq:reconstruction_isolation}
%\end{aligned}\end{equation}
%where $\sum_\text{charged}\pt$ is the scalar sum of the transverse momenta of the charged hadrons, electrons, and muons from the primary vertex located in a cone centered around the lepton direction of size $\Delta R = \sqrt{(\Delta\eta)^2+(\Delta\phi)^2}$ of 0.4 for electrons and muons and 0.5 for tau leptons.
%The sums $\sum_\text{neutral}\pt$ and $\sum_{\gamma} \pt$ represent the same quantities for neutral hadrons and photons, respectively. In the case of electrons and muons the innermost region is excluded
%to avoid the footprint in the calorimeter of the lepton itself from entering the sum.
%Charged particles close to the direction of the electrons are excluded as well, to prevent tracks originating from the conversion of photons emitted by the bremsstrahlung process from spoiling the isolation. In the case of \Tau, the particles used in the reconstruction of the lepton are excluded. The contribution of pileup photons and neutral hadrons
%is estimated from the scalar sum of the transverse momenta of charged hadrons from pileup vertices in the isolation cone $\sum_\text{charged, pileup}$. This sum is multiplied by a factor of 0.5 that approximately corresponds to the ratio of neutral-to-charged hadron production in the hadronization process of inelastic $\Pp\Pp$ collisions. In the case of \Tau, a value of 0.46 is used, as the neutral hadron contribution is not used in the computation of $I_{\tauh}$. An $\eta$, \pt, and lepton-flavor dependent threshold on the isolation variable is applied.


\subsection{Jet and MET reconstruction}
\label{sec:jetmet}
Particle flow algorithm is used to reconstruct jets and \ETmiss. The medium working point of the Combined Secondary Vertex algorithm is used to tag b-jets.

The minimum $\Delta\phi$ between \ETmiss and jets, hereafter referred to as \mindphifour, is used in this analysis to suppress QCD events. To calculate \mindphifour, all the pf-jets in $|\eta|<5.0$ region and with $\pt>40\GeV$ are used without applying any extra identification. $\mindphifour > 1.0$ is found useful for this analysis.

