\section{Physics Object Definition and Preselections}
\label{sect:objdef}


\subsection{Electron identification and reconstruction}
Electron identification is based on a MVA (Boosted Decision Tree) method ~\cite{Hocker:2007ht} which its training takes care of cleaning of electrons collection from $jet \rightarrow e$  fakes. The training process takes advantage of two bins of $\pT$ and three bins of $\eta$, running over a sample of $Z \rightarrow ee$  events selected in data. Those oppositely charged electrons pair closest to Z-boson peak mass is taken as "signal" and the other electron candidates as "background". The training input variables are described in ~\cite{CMS_AN_2013-188}. Different discriminators is trained on electrons which pass the criteria of single electron trigger and for an inclusive sample of electrons, which the second one is chosen.
According on the BDT output,it can be defined a loose and tight working-points for an electron in different bins of $\pT$ and $\eta$ which is given in Table ~\ref{Tab.electronMVAIDwp}.

\begin{table}[!h]
\begin{center}
\begin{tabular}{|l|l|c|c|}
\hline
\multicolumn{2}{|c|}{Kinematic range}                & Loose & Tight \\
\hline
$P_{T} < 20$~\GeV & $\vert \eta \vert < 0.8$         & 0.925 & 0.925 \\
                  & $0.8 < \vert \eta \vert < 1.479$ & 0.915 & 0.915 \\
                  & $\vert \eta \vert > 1.479$       & 0.965 & 0.965 \\
$P_{T} > 20$~\GeV & $\vert \eta \vert < 0.8$         & 0.905 & 0.925 \\
                  & $0.8 < \vert \eta \vert < 1.479$ & 0.955 & 0.975 \\
                  & $\vert \eta \vert > 1.479$       & 0.975 & 0.985 \\
\hline
\end{tabular}
\end{center}
\caption{
  Discriminator thresholds for the Loose and Tight MVA electron identification working--points respectively.
}
\label{Tab.electronMVAIDwp}
\end{table}     

To reject those electrons which is coming from photon conversions, it is demanded that the electron hits each layer of Pixel detector. Furthermore if there is an oppositely charged track near electron which is fitted to a common vertex inside of tracker volume. 


\subsection{Muon identification and reconstruction}


Muons are required to be reconstructed by the Tracker or the Global muon reconstruction algorithm and to be identified as 
muons by tight particle flow algorithm.

The particle flow algorithm identifes  muons by separating them in the several step.


$\bullet$  the number of pixel hit(s) associated to muon track$\geq $1

$\bullet$  Cut on number of tracker layers with hits $\geq $ 6

$\bullet$ the number of hit in muon system $\geq $ 1

$\bullet$ Tracker track matched with at least one muon segment (in any station)

$\bullet$ $\chi ^{2}/NDoF $ for the global track fit$< $ 10 ,

$\bullet$ Impact parameter constrains between the muon track and the selected primary vertex 
 $d_{z} < $0.5 cm and $d_{0} <$ 0.2 cm.

\subsection{Muon and electron isolation}

In order to reduce the background contributions from QCD multi–jet events, electrons and
muons are required to be isolated. The isolation is computed as the $P_{T}$ sum of charged particles (including charged hadrons, 
electrons and muons), neutral hadrons plus photons reconstructed by the PF algorithm within a cone of size
$\Delta R_{iso}$ = 0.4 around the direction of the muon respectively electron. 


In the innermost region
(``veto cone'') neutral
hadrons and photons  are excluded from the computation of the isolation $P_{T}$ sum in order to prevent energy deposits in the electromagnetic and hadronic
calorimeters. Charged particles close to the direction of electrons  are excluded too in order to avoid tracks due to conversions of photons emitted by Bremsstrahlung processes to spoil the isolation.

For the muon isolation the photon and neutral hadron candidates are required to have a transverse energy of MET $>$ 0.5 GeV. This reduces pile–up effects. The correction of pile–up to isolation is done by applying $\Delta\beta$ corrections.
\begin{linenomath}
\begin{equation}
\label{eq:muonisolation1}
I_{e/ \mu}=\Sigma P_{T}^{charged }(\Delta z <2mm)+max(P_{T}^{photon}+P_{T}^{h0 }-\Delta \beta,0),
\end{equation}
\end{linenomath}
The $\Delta\beta$ corrections are computed by summing the transverse momenta of charged particles
that have longitudinal impact parameters $\Delta z > 2$ mm with respect to the lepton production
vertex and scaling the sum by a factor 0.5:
\begin{linenomath}
\begin{equation}
\label{eq:muonisolation2}
\Delta\beta=0.5 \Sigma P_{T}^{charged }(\Delta z <2mm).
\end{equation}
\end{linenomath}

Tau
Nadjieh
