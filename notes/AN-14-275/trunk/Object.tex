\section{Physics Object Definition and Preselections}
\label{sect:objdef}

Electron
Eskandari


\subsection{Muon identification and reconstruction}


Muons are required to be reconstructed by the Tracker or the Global muon reconstruction algorithm and to be identified as 
muons by tight particle flow algorithm.

The particle flow algorithm identifes  muons by separating them in the several step.


$\bullet$  the number of pixel hit(s) associated to muon track$\geq $1

$\bullet$  Cut on number of tracker layers with hits $\geq $ 6

$\bullet$ the number of hit in muon system $\geq $ 1

$\bullet$ Tracker track matched with at least one muon segment (in any station)

$\bullet$ $\chi ^{2}/NDoF $ for the global track fit$< $ 10 ,

$\bullet$ Impact parameter constrains between the muon track and the selected primary vertex 
 $d_{z} < $0.5 cm and $d_{0} <$ 0.2 cm.

\subsection{Muon and electron isolation}

In order to reduce the background contributions from QCD multi–jet events, electrons and
muons are required to be isolated. The isolation is computed as the $P_{T}$ sum of charged particles (including charged hadrons, 
electrons and muons), neutral hadrons plus photons reconstructed by the PF algorithm within a cone of size
$\Delta R_{iso}$ = 0.4 around the direction of the muon respectively electron. 


In the innermost region
(``veto cone'') neutral
hadrons and photons  are excluded from the computation of the isolation $P_{T}$ sum in order to prevent energy deposits in the electromagnetic and hadronic
calorimeters. Charged particles close to the direction of electrons  are excluded too in order to avoid tracks due to conversions of photons emitted by Bremsstrahlung processes to spoil the isolation.

For the muon isolation the photon and neutral hadron candidates are required to have a transverse energy of MET $>$ 0.5 GeV. This reduces pile–up effects. The correction of pile–up to isolation is done by applying $\Delta\beta$ corrections.
\begin{linenomath}
\begin{equation}
\label{eq:muonisolation1}
I_{e/ \mu}=\Sigma P_{T}^{charged }(\Delta z <2mm)+max(P_{T}^{photon}+P_{T}^{h0 }-\Delta \beta,0),
\end{equation}
\end{linenomath}
The $\Delta\beta$ corrections are computed by summing the transverse momenta of charged particles
that have longitudinal impact parameters $\Delta z > 2$ mm with respect to the lepton production
vertex and scaling the sum by a factor 0.5:
\begin{linenomath}
\begin{equation}
\label{eq:muonisolation2}
\Delta\beta=0.5 \Sigma P_{T}^{charged }(\Delta z <2mm).
\end{equation}
\end{linenomath}

Tau
Nadjieh
