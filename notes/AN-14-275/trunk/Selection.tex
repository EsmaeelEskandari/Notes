\section{Event Selection}
\subsection{Search Strategy}
\label{sect:cuts}
In this analysis, two leptons are created with two missing particles, so \mttwo can be a good discriminator to separate signal 
from the SM backgrounds that leptons are produced from W or Z boson. When the mass difference between the chargino and neutralino 
is sufficiently large, \mttwo can exceed 80 GeV which is the maximum of the \mt of a lepton which comes from the decay of a W boson.
When the mass difference is not sufficiently large, \mttwo of the signal events is below 80 GeV and the signal is buried under the W+jets
backgrounds. In such conditions, $\Sigma\,\mt$ which is defined as $\mt_{,\ell1} + \mt_{,\ell2}$ can be useful to distinguish between the signal and 
SM backgrounds.
To optimize the cuts, two signal points are selected, one with a high mass difference ($m_{\chipm}$ = 380\,\GeV and $m_{\chiz}$ = 1\,\GeV) and
another one with a low mass difference ($m_{\chipm}$ = 180 GeV and $m_{\chiz}$ = 60 GeV). An optimized cut should minimize the signal strength, 
which is the ratio of the measured upper limit on the cross section and the theoretical signal cross section. The details of the statistical 
method can be found in section \ref{sect:stat}.

