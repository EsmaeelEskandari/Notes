\section{Backgrounds}
\label{sect:bkg}
In the lepton-Tau channels, the main background is the $W$+jets, when the $W$ boson decays to a lepton and a jet fakes a hadronic tau.
We use a fake rate method to estimate this background. The fake rate is defined as the ratio of the tightly selected taus to the loosely 
selected taus in a sample which is dominated by the fake taus. The fake rate is estimated in an environment which is as similar as possible to 
the signal region. The datasets and the triggers which are used to estimate the fake rate in different channels are shown in 
table \ref{Tab.DataFR}.
\begin{table}[!htb]
\begin{center}
\caption{The datasets and triggers for fake rate estimation.}
\label{Tab.DataFR}
\begin{tabular}{|l|c|c|}
\hline
Channel      & Data Set                                     & Trigger \\\hline
Muon Tau     & /SingleMu/Run2012D-22Jan2013-v1/AOD          & HLT\_IsoMu24\_v(16-17)\\
             &                                              & HLT\_IsoMu24\_eta2p1\_v(14-15)\\\hline
Electron Tau & /SingleElectron/Run2012D-22Jan2013-v1/AOD    & HLT\_Ele27\_WP80\_v11\\
\hline
\end{tabular}
\end{center}
\end{table}

%In the muTau channel, exactly one muon is required which passes the selection criteria of the muon in the signal region and has \pT > 27 GeV. 
Lepton selection and extra lepton rejections are exactly same as the signal selection. Only the \pT of the favorite lepton is forced to 
be 3 GeV/$c$ higher than the online cut (\pT > 27 GeV/$c$ for muon and \pT > 27 GeV/$c$ for electron).
The rejections can reduce the contribution of the $VV$, DY and $t\bar{t}$ events. To further suppress the $t\bar{t}$ contamination, the b-veto 
similar to the signal selection is applied. The \met is asked to be greater than 30 GeV, similar to the preselections.
To avoid any bias from the trigger 


