\section{Backgrounds}
\label{sect:bkgLeptau}
In the $e/\mu-\hadtau$ channels, the main background is the W +jets, when the $\W$ boson decays to a lepton and a jet fakes a hadronic $\tau$.
We use a fake rate method to estimate this background \cite{CMS_AN_2010-261}. 
The idea is that when the loose signal selection is applied, the number of the loose $\hadtau$'s (L) is:
\begin{equation}
L = P + F
\end{equation}
P is the number of the  prompt $\hadtau$'s and F is the number of the  fake $\hadtau$'s. If the selection is tightened, the number of the tight $\hadtau$'s (T) is:
\begin{equation}
 T = pP + fF
\end{equation} 
p (f) is the prompt (fake) rate, the probablity that a loosely selected prompt (fake) $\hadtau$ can pass the  tight  selection. The loose category (L) can be divided to two parts, 
tight (T) and non-tight (NT), so one can write:
\begin{equation}
   F * (f - p) = ((1 - p) * L - NT)
\end{equation}
f * F is the contamination of the fake $\hadtau$'s in the signal region. 

The fake rate ({\it f}) is measured as the ratio of the tightly selected $\hadtau$'s to the loosely 
selected $\hadtau$'s in a sample which is dominated by the fake $\hadtau$'s. The fake rate is estimated in an environment which is as similar as possible to 
the signal region. The datasets and the triggers which are used to estimate the fake rate in different channels are shown in 
table \ref{Tab.DataFR}.
\begin{table}[!htb]
\begin{center}
\caption{The datasets and triggers for fake rate estimation.}
\label{Tab.DataFR}
\begin{tabular}{|l|c|c|}
\hline
Channel      & Data Set                                     & Trigger \\\hline
Muon Tau     & /SingleMu/Run2012D-22Jan2013-v1/AOD          & HLT\_IsoMu24\_v(16-17)\\
             &                                              & HLT\_IsoMu24\_eta2p1\_v(14-15)\\\hline
Electron Tau & /SingleElectron/Run2012D-22Jan2013-v1/AOD    & HLT\_Ele27\_WP80\_v11\\
\hline
\end{tabular}
\end{center}
\end{table}

%In the muTau channel, exactly one muon is required which passes the selection criteria of the muon in the signal region and has \pt > 27 GeV. 
Lepton selection and extra lepton rejections are exactly same as the signal selection. Only the \pt of the favorite lepton is forced to 
be 3 \GeV higher than the online cut resulting to \pt $>$ 27 \GeV for muon and \pt $>$ 27 \GeV for electron.
The rejections can reduce the contribution of the VV, DY and $\ttbar$ events. To further suppress the $\ttbar$ contamination, the b-veto 
similar to the signal selection is applied. The \MET is asked to be greater than 30 GeV, similar to the preselections. The selection of the $\hadtau$ is 
exactly same as the signal selection, except the $\hadtau$ isolation which is {\it Loose} for the loosely selected $\hadtau$'s and {\it Tight} for the 
tightly selected $\hadtau$'s.
The ratio of these two categories determines the fake rate. To avoid any bias from the trigger, the $\hadtau$'s closer than $\Delta$R = 0.2 to the 
lepton are rejected. 
The fake rate can be measured in the bins of $\hadtau$ \pt and $\eta$, but it is observed that the dependency is very small and can be ignored, 
so we use a single value for the fake rate which is XXXX +- XXX.

The prompt rate (p) is measured in the MC DY events. All of the preselections except the Z-veto and $\hadtau$ isolation are applied. The $\hadtau$ isolation 
is relaxed from {\it Tight} to {\it Loose}. Only the events that a generated $\tau$ decays hadronically are considered. If a reconstructed $\hadtau$ is 
closer than $\Delta R = 0.1$ to the generated $\tau$, it is selected. Among these $\hadtau$'s, the prompt rate is defined as the fraction of the loose $\hadtau$'s 
which are tight. The prompt rate can be measured in the bins of \mttwo, but the statistics in the high \mttwo region which is our favorite 
region is too low that we can not conclude anything about the shape of the prompt rate in this region, so we choose it as a constant value
XXXX +- XX in the whole \mttwo region.



