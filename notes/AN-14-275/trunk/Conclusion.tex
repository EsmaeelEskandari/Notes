\section{Conclusion}
\label{sect:conclusion}
A search for SUSY in $\tau\tau$ final state is presented. The $\tau$ pair is produced in a cascade from the production of the \PSGcpDo pair.
Different channels and search bins are introduced to increase the sensitivity to different parts of the phase space. 
Backgrounds and their systematics are discussed in details. 
The expected exclusion limits are also presented for different combination of the channels.
All channels considered have at least one hadronic $\tau$ decay.
There is no excess of events with respect to the SM expectations.
%The other channels were investigated, but they do not add any axclusion power to the analysis.
%Backgrounds and their systematic uncertainties are discussed in details. 
%The expected exclusion limits are also presented for different combination of the channels.
In the context of simplified models, charginos lighter than 420 \GeV 
for a massless neutralino  are excluded at 95\% confidence level.
The limits are also provided for the direct stau pair production, but no masses  can be excluded 
even for a massless neutralino.


\section{Acknowledgments}
This analysis benefits highly from the computing resources of T2 at UCSD and codes developed in ETHZurich. 
We appreciate their help and generosity.
The authors would like to thank the previous and current conveners of the SUSY and SUSY-TBT working group, Frank Wuerthwein, Keith Ulmer, Filip Moortgat, Joshua Thompson, Pieter Everaerts and Boris Mangano for their help and support. 
The authors would like to thank the management and staff of the school of particles 
and accelerators of IPM, especially Prof. Arfaei for their help and support. 
We had several discussions with our colleagues at LIP, Pedram Bargassa, Michele Gallinaro and Cristovao Beirao Da Cruz E Silva. 
We would like to thank them for their helps and suggestions.
The analysis was reviewed inside CMS by Teruki Kamon, as the ARC chair, and Aruna Kumar Nayak, Jan Steggemann and Giovanni Zevi Della Porta. 
The current shape of the analysis has used the constructive and fruitful comments/inputs from ARC members. We thank them deeply.
Thanks to all of the members of
the CMS collaboration for their outstanding results discussed partly here.
