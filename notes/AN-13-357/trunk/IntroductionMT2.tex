\section{Introduction}
\label{sect:introduction}
Supersymmetry \cite{Martin:1997ns} (SUSY) is one of the most promising extensions of the 
Standard Model of the elementary particles (SM) which solves both the 
quadratic divergencies and hierarchy problems simultaneously. It introduces a new symmetry between the bosons and fermions and 
for every particle a sparticle is defined which is exactly the same, but differ in spin by 1/2. 
Since the super particles are not discovered yet, the supersymmetry should be a broken symmetry. Various mechanisms are introduced to 
break the symmetry softly without changing the other interesting features of the theory.

A search for new physics using 20 \invfb of data from CMS taken in 2012 is documented in this note. Although the search is sensitive to any high scale 
new physics with a missing transverse momentum, R-parity conserving SUSY model is used to illustrate the performance of the method.

The search variable is the stransverse mass (\mttwo) which is the natural extension of the known transverse mass (\mt) to a case 
when two massive particles with equal mass are created in pairs and decay via a chain of jets and leptons to two invisible particles. 
In the case of R-Parity conserving SUSY, the Lightest Supersymmetric Particle (LSP) escapes the detection and appears as 
a missing transverse momentum.
The distribution of \mttwo reflects the scale of the produced particles and is much higher for sparticles
compared to the SM particles. Hence, SUSY should appear as an excess in the tail of the \mttwo distribution.
It was shown previously \cite{MT2_2011} that \mttwo is a powerful variable to search for SUSY. Due to consistency of the data with background 
only hypothesis, low mass gluino and squarks have been ruled out. A main direction suggested by the theoreticians and phenomenologists is to 
search for the third generation of the sparticles. % [Arkani Hamed]. 
Since the third generation of the SM particles are heavier than the first two generations, 
in the SUSY sector, this generation can be much lighter. The current analysis is optimized to search for the direct production of 
the supersymmetric partner of the top quark (stop) in the hadronic final states. It is assumed that the pair produced stops undertake the 
following decay chain:
\begin{linenomath}
\begin{equation}
\tilde{t} \rightarrow t + \tilde{\chi_{1}^{0}}
\end{equation}
\end{linenomath}
when top decays hadronically:
\begin{linenomath}
\begin{equation}
t \rightarrow b + W \rightarrow b + q + q'
\end{equation}
\end{linenomath}
and $\tilde{\chi_{1}^{0}}$ can not be detected and appears as missing transverse momentum (\met).

The previous version of the analysis which used only 5.1 \invfb of 2012 data was documented in another analysis note \cite{AN5Invfb}. 
In this version, the full 2012 data is used and some parts of the analysis have been modified.

After introduction in the next section the \mttwo variable is introduced. 
A sepcial method for top reconstruction is described in section \ref{sect:top}.
The data and MC samples are defined in section \ref{sect:dataMC}. 
Different physical objects used in this analysis are introduced in section \ref{sect:objdef}. 
Sections \ref{sect:trigger}-\ref{sect:cuts} review the procedure to 
select the trigger and cuts to have a better reach in this search.
Our strategy to search for stop is explained in section \ref{sect:search}.
Data driven methods are used to estimate the contribution of the main SM backgrounds. 
Section \ref{sect:bkg} shows the methods and their performance.
The statistical methods are used to interpret the results in section \ref{sect:stat} and finally section \ref{sect:conclusion} concludes the note.



