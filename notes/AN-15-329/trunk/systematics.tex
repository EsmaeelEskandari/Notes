The measurement of the cross section is affected by various sources of systematic uncertainties which have to be considered in the fit described in the previous section. To account for this, the effect of systematic uncertainties is evaluated by performing pseudo-experiments. A fit to the $\absetalj$ distribution is performed with the systematic varied shape and the result with respect to the nominal one is taken as the corresponding uncertainty.

The following sources of systematic uncertainties are considered in the analysis:
\begin{itemize}
\item \textbf{Jet energy scale (JES)}: all reconstructed jet four-momenta in simulated events were simultaneously varied according to the $\eta$ and $\pt$-dependent  uncertainties on the jet energy scale~\cite{Chatrchyan:2011ds}. This variation in jet four-momenta is also propagated to $\MET$.
\item \textbf{Jet energy resolution (JER)}: a smearing is applied to account for the known difference in jet energy resolution with respect to data~\cite{Chatrchyan:2011ds}, increasing or decreasing the extra resolution contribution by its uncertainty.
\item \textbf{Muon trigger and reconstruction}: single-muon trigger efficiency and reconstruction efficiency are estimated with a ``tag and probe'' method from Drell--Yan data. The uncertainty on such efficiencies has been conservatively taken in a way to cover the effect of the different kinematics 
in Drell--Yan data and in our single top quark enriched samples, as well as the dependence of the efficiencies from pileup events. 
%\item \textbf{$\rm Q^2$ scale uncertainty}: the uncertainties due to variations in the renormalization and factorisation scale are studied with dedicated simulated samples for $\ttbar$ events and with information provided by the a\MCATNLO generator for $\wjets$ events. These samples and information are generated doubling or halving the renormalisation and factorisation scale with respect to the nominal values. 
\item \textbf{$\rm Q^2$ scale uncertainty}: the uncertainties due to variations in the renormalization and factorisation scale are studied for the signal process, $\ttbar$ and $\wjets$ by reweighting the distributions with different combinations of halved/doubled factorization and renormalization scales.
\item \textbf{Background normalization}:
\begin{itemize}
\item $\ttbar$, tW: $\pm 20\%$, which covers the difference between~\cite{Kidonakis:2012db} and~\cite{Cacciari:2011hy}.
\item \wjets, Drell-Yan: both $\pm 30\%$.
\item $\QCD$ multijet: the normalization is determined from a separate fit to data and is varied by 50$\%$. 
\end{itemize} 
\item  \textbf{b tagging}:  b tagging and misidentification efficiencies are estimated from control samples in 13~TeV data. Scale factors are applied to the simulated samples to reproduce efficiencies in data and the corresponding uncertainties are propagated as systematic uncertainties.
\item \textbf{Signal generator}: the results obtained by using the nominal a{\MCATNLO} signal samples are compared with the result obtained using 
signal samples generated by {\POWHEG} to account for systematics due to different NLO subtraction schemes.
\item \textbf{Luminosity}: the luminosity is known with a relative uncertainty of $\pm \lumiunc$~\cite{lumi}.
\item \textbf{PDF}: the uncertainty due to the choice of the parton distribution functions (PDF) is estimated using reweighted histograms derived from 102 sets of NNPDF.
\item \textbf{MET}: the effect on \MET of a 10\% uncertainty on the unclustered energy deposits in the calorimeters is estimated after subtracting from \MET all jets and leptons.
\item \textbf{Pileup}: the uncertainty on the average expected number of pileup interactions ($\pm 5 \%$) is propagated as systematic uncertainty to this measurement.
\end{itemize}
%\end{itemize}
Table~\ref{tab:systematics} summarizes the different contributions to the cross section uncertainty.
 \begin{table} 
 \topcaption{Relative impact of systematic uncertainties with respect to the observed cross section value $\sigma_{t{\rm -ch}}^{\rm{obs}}$, given in percent.}
 \centering
 \begin{tabular}{ |l|c| } 
  \hline 
Uncertainty source  & $\Delta \sigma_{t{\rm -ch}}/\sigma_{t{\rm -ch}}^{\rm{obs}}$\\
\hline 
 statistical uncertainty & 35.6\% \\
\hline 
JES & 16.8\%\\
JER & 1.1\%\\
b-tagging & 5.6\% \\
lepton reconstruction/trigger & 3.4\% \\
QCD extraction & 1.1\% \\
signal modeling  & 1.9\% \\
factorization and renormalization scales ($Q^2$)  & 3.3\% \\
PDF uncertainty  & 4.5\% \\
MET  & 1.2\% \\
pileup  & 1.4\% \\
 \hline
total systematic uncertainty  & 19.1\%\\
 \hline
luminosity  & 12.0\% \\
 \hline
 \hline
total uncertainty  & 42.0\%  \\
 \hline
%  \hline 
% Uncertainty source  & $\Delta \sigma_{t{\rm -ch}}$ (pb)\\
% \hline 
%  stat. uncertainty & $\pm$ 97.6 \\
% \hline 
% JES & $\pm$  45.9\\
% JER & $\pm$  3.1\\
% b-tagging & $\pm$ 15.3 \\
% lepton reconstruction/trig. & $\pm$  9.2 \\
% QCD extraction & $\pm$  3.0 \\
% signal modeling  & $\pm$  5.1 \\
% fact.renorm scales ($Q^2$)  & $\pm$  9.1 \\
% PDF uncertainty  & $\pm$  12.2 \\
% MET  & $\pm$  3.3 \\
% pileup  & $\pm$  3.9 \\
%  \hline
% total systematics  & $\pm$  52.2\\
%  \hline
% luminosity  & $\pm$ 26.0 \\
%  \hline
%  \hline
% total uncertainty  & $\pm$ 113.7  \\
%  \hline
 \end{tabular} 
 \label{tab:systematics}
 \end{table} 











