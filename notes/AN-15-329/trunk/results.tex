%%%%%%%%%%%%%%%%%%%%%%%%%%%%%%%%%%
%%%%% etalq




%\begin{eqnarray}
%\sigmattop = \xsecobstop \xsecobstopstat \mathrm{(stat.)} \xsecobstopsyst \mathrm{(syst.)},  
%\end{eqnarray}
%\begin{eqnarray}
%\sigmatantitop = \xsecobsantitop \xsecobsantitopstat \mathrm{(stat.)} \xsecobsantitopsyst \mathrm{(syst.)}.
%\end{eqnarray}
%
%From this two separate measurements the ratio $R = \sigmattop/\sigmatantitop$ can be calculated: $R = \Ratio \pm \Ratiouncertstat \pm \Ratiouncertsyst$.

Using the entire sample to measure the inclusive single top cross section (top quark and top anti quark) yields:
\begin{eqnarray}
\xsecRes
\end{eqnarray}


%Figure~\ref{fig:finalplot} shows the comparison of this measurement with the Standard Model expectation and measurements of the single top quark $t$ channel cross section at other center-of-mass energuies.

Such result is used to determine the absolute value of the CKM matrix element $\abs{\mathrm{V_{tb}}}$, assuming that the other relevant matrix elements ($\abs{\mathrm{V_{td}}}$ and $\abs{\mathrm{V_{ts}}}$) are much smaller than$\abs{\mathrm{V_{tb}}}$:
\begin{eqnarray}
\mathrm{\abs{f_{LV}V_{tb}}} = \sqrt{\frac{\sigma_{\mathrm{t-ch.}}}{{\sigma^{\mathrm{th}}_{\mathrm{t-ch.}}}}},
\end{eqnarray}
where $\sigma_{\mathrm{t-ch.}}^{\mathrm{th}}$ is the SM predicted value assuming $\abs{\mathrm{V_{tb}}} = 1$. Possible presence of an anomalous Wtb coupling is taken into account by the anomalous form factor $\mathrm{f_{LV}}$, which is 1 for the SM and deviates from 1 for BSM:
\begin{eqnarray}
\mathrm{\abs{f_{LV}V_{tb}}} = \vtbobs \pm \vtbobsexp (\mathrm{exp.}) \pm \vtbobstheo (\mathrm{theo.}),
\end{eqnarray}
where the first uncertainty contains all uncertainties on the cross section measurement, and the second uncertainty is the uncertainty on the theoretical SM prediction.
The observed (expected) significance is 3.52 (2.70).
